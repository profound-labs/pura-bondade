\chapter{Brahma Viharas}

{\centering\sectionFont\sectionSize
Cântico: Reflexões Sobre o Bem-Estar Universal
\par}

{\itshape
\setlength{\parskip}{0.5\baselineskip}%
\setlength{\parindent}{0pt}%

Possa eu permanecer no bem-estar, livre de aflições, livre de hostilidade,
livre de má vontade, livre de ansiedade; e possa eu manter o bem-estar em mim.

Possam todos permanecer no bem-estar, livres de hostilidade, livres de má
vontade, livres de ansiedade; e possam todos manter o bem-estar neles
próprios.

Possam todos os seres libertar-se de todo o sofrimento.

E possam não ser separados da boa fortuna que alcançaram.

Quando eles agem com intenção , todos os seres são donos da sua acção e herdam
os seus resultados. O seu futuro nasce dessa acção, é companheiro dessa acção,
e os seus resultados serão o seu lar. Todas as acções intencionais, sejam
cuidadosas ou danosas - destes actos eles serão os herdeiros.

}

\bigskip

Este é o cântico que trata do que denominamos de Brahma Viharas:
benquerença, compaixão, alegria empática e equanimidade ou serenidade. A
primeira parte do cântico é sobre desejar bem a nós próprios e depois
aos outros, na segunda deseja-se que todos os seres sejam libertos de
sofrimento, na terceira que não se separem da boa fortuna que alcançaram
e que possam desfrutar realmente dos sucessos que têm nas suas vidas. A
quarta parte é uma reflexão sobre o \emph{Kamma}, que pode ser muito
útil quando ocorrem certas coisas nas nossas vidas. Dá-nos a
possibilidade de aceitá-las simplesmente como são, reconhecendo que são
o resultado do que ocorreu antes. Compreendemos igualmente que a forma
como reagimos às coisas no momento presente irá afectar a forma como as
coisas nos irão correr no futuro.

