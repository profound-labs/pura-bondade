\chapter{Brahma Viharas}

{\centering\sectionSize
Cântico: Reflexões Sobre o Bem-Estar Universal
\par}

% \itshape every par for the markdown converter

{\setlength{\parskip}{0.5\baselineskip}%
\setlength{\parindent}{0pt}%

\itshape
Que eu mantenha bem-estar,\\
Livre de aflição,\\
Livre de hostilidade,\\
Livre de má-fé,\\
Livre de ansiedade,\\
E possa eu manter em mim bem-estar.

\itshape
Que todos mantenham bem-estar,\\
Livres de hostilidade,\\
Livres de má-fé,\\
Livres de ansiedade, e possam eles\\
Manter bem-estar em si próprios.

\itshape
Possam todos os seres se libertarem de todo o sofrimento.

\itshape
E que todos não se separarem da boa fortuna que alcançaram.

\itshape
Quando agem com intenção,\\
Todos os seres são os donos de sua acção e herdam seus resultados.\\
O seu futuro nasce de tal acção, companheiro de tal acção,\\
E os seus resultados serão o seu lar.

\itshape
Todas as acções com intenção,\\
Sejam elas boas ou más --\\
De tais actos eles serão os herdeiros.

}

\bigskip

Este é o cântico que trata do que denominamos de Brahma Viharas:
benquerença, compaixão, alegria empática e equanimidade ou serenidade. A
primeira parte do cântico é sobre desejar bem a nós próprios e depois
aos outros, na segunda deseja-se que todos os seres sejam libertos de
sofrimento, na terceira que não se separem da boa fortuna que alcançaram
e que possam desfrutar realmente dos sucessos que têm nas suas vidas. A
quarta parte é uma reflexão sobre o \emph{Kamma}, que pode ser muito
útil quando ocorrem certas coisas nas nossas vidas. Dá-nos a
possibilidade de aceitá-las simplesmente como são, reconhecendo que são
o resultado do que ocorreu antes. Compreendemos igualmente que a forma
como reagimos às coisas no momento presente irá afectar a forma como as
coisas nos irão correr no futuro.

