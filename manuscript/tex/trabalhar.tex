\chapter{Trabalhar Os Obstáculos}

Há uns dias, quando fui passear na floresta, estava uma árvore enorme
caída no caminho, o que dificultava muito a passagem. Esta manhã fui lá
outra vez e fiquei muito contente pois um tractor tinha removido a
árvore. Tive a possibilidade de ir até mais longe na floresta. É isto
que se passa com as nossas mentes no decurso da prática. Ocasionalmente,
existe um obstáculo muito grande, um problema muito grande, e podemos
necessitar de recorrer a profissionais - por vezes as pessoas sentem que
fazer psicoterapia ou utilizar uma determinada técnica as pode ajudar a
enfrentar e trabalhar uma dificuldade específica. Contudo, num retiro,
em geral, temos de enfrentar e trabalhar com obstáculos mais pequenos e,
alguns deles, desaparecem por si próprios - com atenção benévola,
aceitação e muita paciência, as coisas podem mudar. Por vezes as
mudanças ocorrem apenas com uma ligeira mudança de atitude. De igual
modo, com frequência, existem coisas que podemos fazer como uma
estratégia deliberada, de forma que o obstáculo seja completamente
removido ou que, pelo menos, possamos trabalhar com ele de uma maneira
mais positiva. Gostava de partilhar alguns dos conselhos do Buddha
relativos ao trabalho com diferentes tipos de obstáculos.

Se existe uma disposição particularmente negativa relativamente a alguém
ou a alguma situação, o Buddha sugeriu que subtituíssemos isso por uma
atitude diferente. É muito difícil estarmos zangados com alguém quando
paramos para ter em conta a sua situação, qual a intenção que essa
pessoa poderia ter tido. Por vezes, o simples acto de nos colocarmos na
pele da outra pessoa pode ajudar a largarmos a zanga ou negatividade que
temos face a ela. Este é um tipo de \emph{metta} ou de prática de
benquerença, que é igualmente útil quando nos sentimos frustrados ou
preocupados com a nossa prática ou com o que está a ocorrer na nossa
mente. Talvez tenhamos o hábito de ser muito duros e críticos connosco
próprios e portanto podemos em vez disso experimentar ser benevolentes,
compreensivos e encorajadores.

Levei muito tempo até perceber quão crítica sou para mim mesma. Por
achar esta uma prática muito benéfica decidi, no ano novo, abandonar a
auto depreciação, decidi deixar de ser crítica e de pensar mal sobre mim
própria. Isto foi muito útil, porque nunca me tinha apercebido o quanto
era autocrítica, o quanto me criticava tanto a mim própria. Tive
bastante oportunidade para praticar. Logo que me apercebia de um
pensamento do género «Não fizeste aquilo muito bem», pensava
deliberadamente «Não. Não vás por aí. Não penses assim.» Evidentemente,
tive de fazer isto repetidamente, mas foi espantoso como, passadas
apenas uma ou duas semanas, comecei a sentir-me muito mais contente e
leve.

Desta forma, temos de reconhecer, pacientemente, o pensamento negativo
ou hostil. Por vezes as pessoas vêm ter comigo nos retiros e dizem: «É
terrível. Nunca me tinha apercebido de como era negativo e hostil.»
Então digo-lhes: «Não se preocupe. Na realidade, é muito bom ter
reconhecido pois agora tem a possibilidade de fazer algo em relação a
isso.» Era isto que o Buddha queria dizer com substituir um pensamento
com o pensamento contrário. A parábola que utilizou para isto foi a de
um carpinteiro que utiliza uma cavilha mais pequena para extrair outra
cavilha maior da madeira. Talvez isto não faça sentido para quem não
tenha conhecimentos de carpintaria - a ideia é que as duas cavilhas não
conseguem estar no mesmo buraco em simultâneo. Assim, este é o primeiro
método que o Buddha recomendou para lidar com um obstáculo e podemos
utilizá-lo para qualquer obstáculo.

Outra estratégia sugerida pelo Buddha consiste em sentir verdadeiramente
quão desagradável é um estado de espírito negativo. Isto pode ser algo
que reconheçamos em nós próprios, se nos encontramos a atravessar uma
fase em que estamos mesmo maldispostos e sempre a reclamar; ou pode ser
mais fácil de observar noutra pessoa. A maior parte de vocês,
provavelmente, conhece alguém que está sempre a queixar-se de tudo, que
consegue ver sempre os aspectos negativos ou falhas em qualquer
situação. O Buddha encorajou-nos a contemplar o quão desagradável é
viver desta forma. Sei por experiência própria que, quando dei ouvidos a
estas vozes a reclamarem na minha própria cabeça e experienciei
realmente como as sentia, compreendi que não tinha necessidade de pensar
desta forma até ao fim da minha vida e também que não preciso de pensar
desta forma agora. A parábola que o Buddha utilizou para esta situação é
bastante chocante. Ele disse que é como uma pessoa muito bonita a usar
um colar feito de cadáveres de cães -- completamente desnecessário e
verdadeiramente repulsivo. Isto pode ser um pouco forte mas pode alertar
a mente para este tipo de estado, e para o facto de termos realmente a
possibilidade de escolher se usamos (ou não) esse colar, ou não.

Outra estratégia é, simplesmente, dirigir a mente para algo diferente.
Um exemplo muito óbvio é o seguinte: se temos muita coisa na cabeça,
dirigimos a nossa atenção para a respiração ou para o corpo. Como sugeri
anteriormente, podemos olhar para a mente como um quarto. As coisas e
pessoas no quarto são como os pensamentos, os objectos mentais. Podemos
pensar: «Gosto deste, não gosto deste, este é bom.» Podemos estar muito
ocupados a escolher pensamentos que gostamos e que não gostamos, mas já
repararam em mais alguma coisa no quarto? Para além disso o que é que
este contém? Espaço\ldots{} Por isso, em vez de nos centrarmos nos
objectos, podemos focar-nos no espaço à volta destes. Por vezes quando
pensamos muito, só conseguimos ver esse pensar; é como se ocupasse toda
a mente. Mas existe uma forma de reconhecer que a mente é muito maior do
que o pensar. Em vez de nos centrarmos no pensamento, podemos focar-nos
no espaço em redor do pensamento.

Se temos uma emoção muito forte, como a ira ou o pesar, podemos começar
por pensar nisso e interrogarmo-nos o que fazer, pensando que está
alguma coisa errada, porque estamos a sentir essa emoção, e
interrogarmo-nos como nos iremos ver livres dela. Mas isto aumenta e
reforça a emoção. Em vez disso, por vezes é útil simplesmente
centrarmo-nos no corpo. Com emoções como a ira, a ansiedade, o medo ou o
pesar, existe sempre uma sensação física, que a acompanha, no coração,
na barriga ou no plexo solar. Assim, em vez de sermos apanhados na
situação, no acontecimento, ou seja o que for que tenha despoletado a
reacção emocional, podemos simplesmente trazer a atenção para o corpo e
observar as mudanças, à medida que estas ocorrem no corpo. É uma forma
de nos libertarmos de tal. Em vez de nos agarrarmos à emoção - sendo
levados por ela ou entregando-nos a ela - ou de lutarmos para nos vermos
livres dela, em vez disso, abrimos mão. Desta forma podemos observar
como se modifica.

A quarta estratégia consiste naquilo que chamamos abrandar o processo
mental. Podemos estar a pensar muito, talvez de forma não muito clara e
com muitos pensamentos à mistura, mas esta estratégia envolve pensarmos
deliberadamente mais ainda; é como trazer os pensamentos para o primeiro
plano da mente e olhar cuidadosamente para aquilo que estamos realmente
a pensar. Por vezes pode ser útil escrever os pensamentos; outras vezes
podemos jogar com eles na nossa mente. Se estivermos zangados ou
perturbados, uma das coisas interessantes que, provavelmente, iremos
notar são aqueles pensamentos que só são uma espécie de resmungo na
mente. Não estão articulados de forma clara. Então, podemos dizer para
nós próprios: «Ok, vamos lá a ver o que se está realmente a passar aqui.
Quero ouvir o que estás a dizer.» É como uma criança pequena que está
aborrecida e grita «Uaaaaaah!» e nós perguntamos: «O que foi?» Por
vezes, fazer só isto e dar atenção ao que se diz, é o suficiente para
ajudar a criança a largar a birra.

A técnica final, que o Buddha recomendou e apenas para situações
notoriamente extremas, é a supressão forçada. Ele disse isto como se
existissem dois lutadores, um muito grande e o outro mais pequeno. O
maior imobiliza o outro, de forma que não conseguem mover-se de todo.
Por vezes podemos precisar de fazer isto, mas apenas o podemos fazer
durante pouco tempo, uma vez que requer um grande esforço.

Por exemplo, a emoção da ira pode ser tão forte que sentimos que podemos
realmente praticar um acto violento. Evidentemente que estamos todos a
praticar de acordo com os preceitos, comprometemo-nos a não agredir
ninguém, e assim, quando isto acontece, temos apenas de ficar muito
quietos e silenciosos e direccionar fortemente a mente noutra direcção.
Mais adiante, estando mais calmos, poderá ser útil darmo-nos algum tempo
para considerar por que é que a emoção era tão forte para nos perturbar.
Desta forma podemos obter algum discernimento relativamente ao
sentimento de vulnerabilidade, de ansiedade ou do que quer que seja que
tenha feito emergir uma reacção tão forte e, talvez, encontrar uma forma
de evitar que uma situação semelhante surja novamente. Mas, na altura, a
emoção pode ser tão forte que tenhamos de recorrer a esta medida extrema
de supressão forçada.

Estou certa que todos vocês têm muitas outras formas de trabalhar com a
mente. Estas são apenas algumas sugestões dos ensinamentos do Buddha que
considero úteis.

\section{Meditação Guiada}

Aquietem a mente. Aquietem o corpo.

Apercebam-se que estão a pensar e do que estão a pensar. Se a mente
estiver bastante calma, façam surgir deliberadamente algum pensamento -
pensem talvez no que comeram ao pequeno-almoço ou algo do género,
bastante neutro.

Estejam conscientes do pensamento como um objecto mental, algo na mente
a observar, do qual se conseguem aperceber.

Pode parecer que o pensamento vos esteja a preencher a mente. Agora,
gostaria de vos encorajar a tentar expandir um pouco a mente, a torná-la
maior. Uma forma de o fazer é tomar consciência do espaço à nossa volta,
para além do nosso corpo. Deixem que a mente se expanda no espaço da
sala. Reparem que, quando fazemos isto, existe muito mais espaço em
redor do pensamento.

Então podemos centrar-nos no próprio pensamento, ou podemos colocar a
nossa consciência no espaço em redor do pensamento. Quando fazemos isto,
verificamos que, por vezes, a mente se encontra muito envolvida com o
acto de pensar, mas podemos deixar ir estes pensamentos e expandi-la, de
forma a criar espaço à volta do pensamento.

Experimentem durante algum tempo a consciencializarem-se do pensamento
e, depois, do espaço em redor do pensamento.

(pausa)

Agora tentem encontrar uma frase curta na corrente do pensamento. Se
tiverem dificuldade em fazê-lo pensem simplesmente numa frase, uma frase
curta do género: «Estou a respirar» ou «Gosto desta prática» ou «Não
gosto desta prática». Esta frase pode encher a mente e nós podemos criar
um espaço em redor do pensamento.

Podemos igualmente abrandar o pensamento. Fiquem algum tempo a pensar
cada palavra, com um espaço antes da próxima. Seja qual for a frase -
algo que não tenha grande significado ou qualquer coisa que tenha uma
grande carga emocional - podem sempre fazer esta experiência.

Se tiverem dificuldade em ouvir o pensamento, podem tentar visualizar as
palavras, aumentando-as, diminuindo-as ou com diferentes cores. Se
escolherem cores para as palavras, podem reparar que as cores se alteram
à medida que a intensidade da emoção ou do pensamento começa a diminuir.

À medida que terminamos a meditação, gostava de sugerir que
substituíssem as palavras que têm na mente, sejam elas quais forem, por:
«Que este ser fique bem.» Repitam estas palavras algumas vezes, deixando
que elas encham a mente e o corpo.
