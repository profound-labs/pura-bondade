\chapter{Os Três Refúgios e Os Cinco Preceitos}

As palavras \emph{Buddha}, \emph{Dhamma} e \emph{Sangha} são
frequentemente usadas na nossa prática. A palavra \emph{Buddha}
refere-se ao professor histórico, significando igualmente a nossa
própria capacidade para despertar, a capacidade de ver as coisas
claramente. O \emph{Dhamma} é o ensinamento dado pelo Buddha, que aponta
para a verdade que cada um de nós pode apreciar ou saber por si próprio,
quando nos encontramos completamente presentes ou na realidade tal como
é. O \emph{Sangha} é a comunidade dos discípulos do Buddha - gerações
incontáveis de homens e mulheres, desde o tempo do Buddha, que ouviram
os ensinamentos e que, através da sua aplicação nas suas vidas, têm sido
capazes de compreender a verdade por eles próprios e experimentar paz
nos seus corações. A palavra \emph{Sangha} pode igualmente referir-se à
comunidade de pessoas que se apoiam mutuamente na sua prática, tal como
nós nos apoiamos mutuamente durante este retiro.

Chamamos ao Buddha, ao Dhamma e ao Sangha os \emph{Três Refúgios}. Estes
podem igualmente ser chamados as Três Jóias ou a Gema Tripla. Uma jóia é
algo requintado que não pode ser danificado, o que constitui uma
comparação útil, uma vez que podemos confiar sempre nestes três refúgios
- eles estão sempre presentes, independentemente de onde quer que vamos
e do que façamos. Por isso, encorajo-vos a reflectirem sobre estes
tesouros sem preço, estes refúgios - é desta forma que eles poderão vir
a ter um significado e um valor reais nas vossas próprias vidas.

O apoio tradicional para aqueles que escolhem seguir os ensinamentos do
Buddha são estes três refúgios, juntamente com os Cinco ou os Oito
Preceitos. Os Cinco Preceitos são as directrizes éticas que o Buddha
recomendou e encorajou para os leigos, nas suas vidas do dia-a-dia. Os
Oito Preceitos são para os noviços ou para os visitantes de um mosteiro
- são, de igual forma, geralmente utilizados como apoio num período de
retiro. O Primeiro Preceito envolve a abstenção de matar - não tirar a
vida a nenhum ser, mesmo a criaturas minúsculas que nos desagradam.
Podemos usar este preceito como uma directriz para nós próprios.
Contudo, se acidentalmente fizermos mal a um insecto, apesar de ser
lamentável, não constitui uma quebra deste Preceito. O Preceito apenas
quer dizer que nos devemos abster de, intencionalmente, causar qualquer
mal ou matar qualquer ser vivo. Em segundo lugar abstermo-nos de retirar
seja o que for que não nos tenha sido dado ou disponibilizado para
utilizarmos. Nós não roubamos. Este Preceito é muito útil porque
significa que podemos viver juntos e confiar uns nos outros, o que é
algo bonito.

O Terceiro Preceito é abstermo-nos de qualquer actividade sexual
intencional. Quando inserido nos Cinco Preceitos, traduz-se na abstenção
de qualquer conduta sexual danosa. Devemos compreender que este Preceito
não constitui uma proibição relativa ao desejo sexual - toda a gente
sente, naturalmente, desejo sexual. Aquilo que nos é pedido é para nos
abstermos de agir de acordo com esse desejo ou, no caso dos Cinco
Preceitos, de procurar qualquer forma de gratificação sexual com alguém
com quem não tenhamos uma relação que envolva um compromisso.

O Quarto Preceito trata de nos abstermos do discurso incorrecto. No
contexto dos Cinco Preceitos isto pode ser interpretado como o cultivo
do discurso cuidado e a abstenção dos quatro tipos de discurso danoso: a
mentira, a intriga, a palavra rude ou danosa e o tagarelar frívolo.
Quando fazemos um retiro, em geral é-nos pedido que observemos o Nobre
Silêncio, uma vez que, quando mantemos o silêncio externamente, temos
uma boa oportunidade para escutar o tagarelar interno da mente. Espero
que tenham igualmente oportunidade de observar como esse tagarelar
começa a acalmar, nem que seja apenas um pouco. Alguns de vocês vão
sentir uma grande paz. Para outros podem ser apenas alguns momentos aqui
e ali. Quer se trate de uma grande paz ou apenas de uma pequena paz, não
tem importância. O que importa é estar presente e reparar no que está a
acontecer. Não existe um prémio para a mente que conseguir estar mais em
paz!

O Quinto Preceito envolve a abstenção de tomar qualquer tipo de
intoxicantes - álcool ou drogas recreativas. Se necessitarem de tomar
algum medicamento que vos tenha sido prescrito, óptimo - e tenho o
prazer de vos dizer que a cafeína também é permitida.

O Sexto Preceito envolve a abstenção de comer depois do \mbox{meio-dia}, ou em
alturas inapropriadas. A ideia é não utilizar a comida como uma forma de
distracção. Recordo-me que, quando era leiga, sempre que me sentia um
pouco entediada ou infeliz, lançava automaticamente a mão a qualquer
coisa comestível. Contudo, durante este retiro peço-vos para não fazerem
e, em vez disso, observarem o sentimento, seja ele qual for, de
aborrecimento ou infelicidade que poderão estar a experienciar, bem como
para repararem como é que esse sentimento se altera. Desta forma poderão
ter uma compreensão profunda da impermanência - podem observar,
realmente, por vocês próprios, como as coisas se modificam.

O Sétimo Preceito consiste na abstenção relativamente a entretenimentos,
embelezamento e adornos. Não usamos grinaldas no nosso cabelo ou outro
tipo de jóias (alianças são permitidas) e não participamos em desportos
ou jogos, nem ouvimos música ou praticamos qualquer das actividades que,
em geral, são consideradas divertimento. Em vez disso, temos uma
oportunidade de experienciar um tipo de prazer muito mais subtil - e
divertido - através da nossa meditação.

O Oitavo Preceito consiste na abstenção de nos deitarmos numa cama
elevada ou luxuosa, o que pode soar um pouco estranho. No entanto,
trata-se de um estímulo no sentido do cultivo de uma atitude de vigília,
em vez da utilização do sono como uma fuga ou da procura do prazer
através da posse de mobiliário luxuoso. Isto não significa que não
possamos descansar ou que não devamos dormir; apenas que descansamos na
medida em que isso é necessário. Podemos aprender muito estando
acordados e atentos ao que ocorre nas nossas mentes e nos nossos corpos.

É importante ver estes Preceitos como apoios amigáveis, ao invés de
agentes da polícia secreta que vão estar a vigiar-nos e a \mbox{punir-nos} se
cometermos algum erro. Eles estão disponíveis para serem usados por nós,
nas nossas vidas, se assim escolhermos. Os Cinco Preceitos são um
lembrete para nos mantermos no âmbito de fronteiras éticas, de forma
que, tanto quanto possível, evitemos comportamentos que sejam danosos
para nós ou para os outros - comportamentos que nos metam em apuros. No
retiro, os restantes Preceitos de Renúncia contribuem para a
simplificação. Fazemos uma escolha consciente em não participar em
determinadas actividades. Através desta escolha criamos o espaço dentro
de nós. Podemos então observar directamente as nossas formas habituais
de perceber e de responder às experiências. Desta maneira podemos ver o
que é necessário para libertar o coração do sofrimento.

