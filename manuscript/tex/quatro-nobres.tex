\chapter{As Quatro Nobres Verdades}

Há uma história, que muitos de vocês devem conhecer, que começa com o
Buddha a caminhar na floresta com alguns dos seus discípulos. A certa
altura, o Buddha baixa-se e apanha uma mão cheia de folhas do chão.
Estende as folhas aos discípulos e diz: «Digam-me se existem mais folhas
nas árvores e no chão, ou na minha mão?» Os discípulos dizem: «Existem
apenas algumas folhas na sua mão, mas existem inúmeras folhas nas
árvores e no chão da floresta.» O Buddha responde: «Sim, é verdade. As
folhas nas árvores e no chão representam todas as coisas que o Ser
Perfeito pode saber e as folhas na minha mão representam aquilo que eu
ensino: o que vocês precisam saber e contemplar, de forma a libertarem o
coração do sofrimento.»\footnote{SN 56.31, Simsapa Sutta (As Folhas de
  Simsapa).}

Alguns de vocês poderão gostar de especular sobre o início e o fim do
universo, bem como sobre todo o tipo de questões para as quais não
existe verdadeiramente uma resposta. Contudo, o Buddha encorajou-nos a
não nos preocuparmos com esses assuntos, mas antes, que tivéssemos
atenção a apenas quatro coisas, referindo-se a estas como as Quatro
Nobres Verdades. À~\mbox{primeira} chamou a Nobre Verdade do Sofrimento, a qual
se refere ao facto de nada no mundo condicionado poder proporcionar paz
ou bem-estar duradouros. Em resultado disto, a existência humana é
inerentemente insatisfatória.

Em segundo lugar, temos a Nobre Verdade da Origem do Sofrimento - esta
refere-se à existência de uma razão pela qual sofremos e ao facto de
podermos descobri-la através da nossa própria observação. Quando olhamos
cuidadosamente, podemos reparar como uma sensação de desconforto pode
criar o desejo de querer que as coisas sejam diferentes daquilo que são.
Quando nos apegamos a esse desejo, quando investimos nele, este apego
traz uma sensação de tensão, uma sensação de conflito.

A Terceira Nobre Verdade é a Verdade da Cessação do Sofrimento, do fim
da tensão ou do conflito. Isto surge quando largamos aquele desejo de as
coisas serem diferentes do que são. O desejo, em si próprio, pode
continuar, contudo nós abdicamos ou abandonamos o nosso empenho nele.
Fazemos as pazes com a realidade tal como ela é.

A Quarta Nobre Verdade é a Nobre Verdade do Caminho que Conduz ao Fim do
Sofrimento: consiste nas orientações que o Buddha nos deixou sobre como
viver a nossa vida de forma a sofrermos cada vez menos.

No decurso deste retiro iremos contemplar estas verdades apresentadas
pelo Buddha. Se estamos a sentir sofrimento, podemos dar por nós a
pensar que temos de fazer alguma coisa, que temos de escapar ao
sofrimento, mas o Buddha disse: «Não, o sofrimento deve ser
compreendido.» Não conseguimos compreender o sofrimento nas nossas vidas
se estamos constantemente a tentar fugir dele ou a distrairmo-nos. Por
isso aquilo que proponho é que, ao longo destes dias, se interessem,
sejam curiosos relativamente à vossa vida e experiência - incluindo
mesmo as tensões, as pressões e lutas aparentemente triviais ou as
formas subtis de aversão ou negatividade. Interessem-se verdadeiramente
pela forma mais insignificante de ansiedade ou de medo, como um
cientista a examinar algo ao microscópio.

Há pouco, quando fui dar uma volta a pé, reflecti em como fazer um
retiro é um pouco como estar de férias e fazer amizades. Isto pode
parecer um pouco surpreendente aos que estão habituados ao tipo de
retiros nos quais são encorajados a trabalhar arduamente, nos quais é
enfatizado o esforço despendido para progredir na prática da meditação.
Também podem interrogar-se sobre o que quero dizer com «fazer amizades»
quando praticamos o Silêncio Nobre: não falar uns com os outros a não
ser que seja mesmo necessário. Com efeito, o que eu gostava de encorajar
o mais possível é que alcancem uma sensação agradável, de tranquilidade
e descontracção. Não me refiro ao tipo de relaxamento que leva a que
simplesmente adormeçam, apesar de alguns de vocês poderem sentir muita
sonolência nos primeiros dias. Este tipo de tranquilidade e
descontracção a que me refiro é um estado de vigília, uma inteligência,
e o tipo de amizade que sugiro é que tentem ser mais amigáveis com vós
próprios.

Uma das coisas mais tristes e difíceis para as pessoas na nossa cultura
ocidental parece ser a falta de capacidade de fazerem amizade consigo
próprias, para se aceitarem tal como são. Podemos ser bondosos para com
as outras pessoas, perdoá-las e aceitá-las, mas quando olhamos para as
nossas mentes e para a forma como nos relacionamos connosco próprios,
vemos que, frequentemente, podemos ser muito cruéis, muito exigentes e
severos nos juízos sobre nós mesmos. Por isso encorajo-vos a
conhecerem-se melhor nestes próximos dias, de uma forma mais gentil.
Reparem naquilo que não gostam e que não aprovam em vós próprios, e
vejam apenas se se podem perdoar e aceitar as coisas tal como são. Desta
forma pode dar-se uma verdadeira transformação. Do mesmo modo,
apercebo-me que quanto mais faço amizade comigo própria, mais apta estou
a fazer amizade com outras pessoas.

Se insistimos em reprimir o que não nos agrada, os nossos pensamentos e
as nossas disposições desagradáveis, não nos estamos a aceitar realmente
e provavelmente acabaremos por adoecer. Parece que grande parte das
doenças dos tempos atuais resulta de não cuidarmos adequadamente da
nossa vida emocional.

Por isso, podemos ver este tempo de retiro como uma oportunidade de
fazer algum trabalho preventivo importante. Se criarmos uma atmosfera
interna de confiança e de bondade, os nossos maus hábitos de pensamento
podem revelar-se. Então, tendo visto e admitido estes hábitos, podemos
abrir mão deles, de forma que as nossas vidas deixem de ser limitadas
pelos seus efeitos negativos.

