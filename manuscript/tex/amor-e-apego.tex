\chapter{Amor e Apego}

O tema sobre o qual gostaria de falar agora é «Amor e Apego». Penso que
é um bom tópico, uma vez que, com frequência, os budistas acham ou
sentem que é errado ter apego ou que é errado amar outras pessoas.
Existem mesmo alguns Budistas verdadeiramente académicos que pensam que
não devemos sentir qualquer emoção. Eles pegam na palavra «desapego» e
pensam que quer dizer não sentir seja o que for. Por isso acho
interessante este tema do amor e do apego, porque é óbvio que as pessoas
gostam mesmo umas das outras - eu realmente gosto muito dos meus
familiares e, quando estou a orientar um retiro, começo gradualmente a
gostar de todos os participantes. Desenvolvo um interesse e uma
preocupação reais relativamente ao seu \mbox{bem-estar}.

Há uma história da vida do Buddha que considero muito boa como
ilustração. Em tempos existia um mercador rico e muito influente, que
vivia onde o Buddha estava a viver. O Buddha conhecia-o. Uma manhã o
homem estava a andar na vila e encontrava-se num estado de grande
sofrimento, porque o seu único filho tinha acabado de falecer. A
resposta do Buddha à angústia deste homem pode chocar alguns de vocês.
Ele disse: «Os entes queridos trazem mágoa.»

O mercador irritou-se; não concordava com o que o Buddha tinha dito.
Andou a perguntar às pessoas: «Ouviu isto? Ouviu o que o Buddha disse?
Disse que os entes queridos trazem mágoa. Mas isto não é verdade, de
todo. Eles trazem satisfação. Eles trazem alegria.» Em breve toda a
gente na vila estava a falar sobre este incidente e o assunto chegou
mesmo ao palácio, ao Rei Pasenadi. Por acaso, a mulher do Rei Pasenadi,
a Rainha Mallika, era uma discípula muito dedicada do Buddha. O Rei e a
Rainha falaram acerca do que o Buddha teria dito e o Rei disse: «Acha
que o Buddha disse mesmo aquilo? Será que isso está mesmo correcto?» A
Rainha disse: «Bem, se o Buddha o disse, então deve ser verdade.» Isto
irritou o Rei Pasenadi, que disse: «Se o Buddha afirma alguma coisa,
diz-me sempre que deve ser verdade!» Então a Rainha decidiu enviar um
mensageiro para perguntar ao Buddha o que tinha dito ao mercador. A seu
tempo teve a resposta que, efectivamente, tinha sido isso que o Buddha
tinha dito.

Assim, a Rainha pensou sobre este assunto e foi ter com o marido e
disse: «Sei que todos pensam que os entes queridos nos dão satisfação e
alegria, contudo, como se sentiria se acontecesse alguma coisa à nossa
querida filha?» O Rei disse: «Claro que ficaria aflito se alguma coisa
menos boa acontecesse à nossa linda filha, a Princesa Vajiri.» Então a
Rainha continuou a perguntar sobre todos aqueles de quem o Rei gostava,
como se sentiria se acontecesse alguma coisa a este ou àquele. Por fim o
Rei compreendeu aquilo que o Buddha estava a realçar e aceitou que este
realmente tinha razão.

Evidentemente, se gostamos muito de alguém e existe um forte apego a
essa pessoa, é natural sentirmos pena, quando alguma coisa má lhe
acontece, e alegria quando alguma coisa boa lhe ocorre. Isto é natural,
é normal. Mas como referi anteriormente, alguns budistas acham que nunca
nos devemos apegar aos outros. Por vezes ouço algumas pessoas dizerem:
«Sei que não me devia apegar mas não o consigo evitar, eu gosto mesmo
muito da minha mãe.»

Temos um cântico que entoamos com bastante frequência, intitulado
\emph{Cinco Assuntos para Relembrar com Frequência,} que começa da
seguinte forma:

\begin{quote}
\itshape
É da minha natureza envelhecer.\\
É da minha natureza adoecer.\\
É da minha natureza morrer.\\
Tudo o que é meu, querido e agradável,\\
tornar-se-á diferente, será separado de mim.
\end{quote}

Podemos achar isto terrivelmente deprimente, mas prefiro considerar esta
reflexão como uma forma de desenvolver sabedoria e compreensão. São
aspectos da vida sobre os quais a maior parte das pessoas não gosta
muito de pensar, mas tenho verificado, na minha vida e na minha própria
prática, que tê-los presentes, na realidade, tem-me ajudado a apreciar
as coisas de forma mais plena. Quer se trate de um acontecimento ou de
um relacionamento, esta reflexão tem-me encorajado a estar muito
presente e a apreciar cada momento.

À medida que os meus pais estavam a envelhecer, dei por mim a recear o
dia em que eles faleceriam. Não sabia como é que as pessoas sobreviviam
a este tipo de perda; por isso \mbox{preocupava-me} com o momento futuro em que
eles morreriam. Mas depois percebi que o meu tempo seria talvez melhor
empregue a usufruir do nosso relacionamento enquanto eles eram vivos, em
vez de me preocupar constantemente com a nossa inevitável separação.
Assim, acabei por passar bons momentos com eles, fiz questão de lhes
dizer coisas que me pareciam importantes, uma vez que sabia que chegaria
o tempo em que não poderia dizer-lhes essas coisas da mesma forma.
Depois, evidentemente, quando chegou a hora deles falecerem,
experienciei uma grande tristeza. Ainda me sinto triste, por vezes,
quando penso neles, e por vezes gostava que eles ainda aqui estivessem -
mas eles estavam a ficar cada vez mais velhos e era a altura certa de
partirem. E não me importei com a tristeza; não sofri com ela. Era
apenas tristeza e vi-a como a consequência inevitável de amar alguém. É
quase como se, quando se ama alguém, existisse uma etiqueta com o preço
agarrada. Mas eu estava perfeitamente disposta a pagar o preço. Não era
excessivo.

Por isso encorajo-vos a aprender a amar conscientemente, ao invés de
amar com apego cego e com exigência escondida: «Por favor nunca me
deixes. Por favor não envelheças, não adoeças e, faças o que fizeres,
não morras!» Provavelmente, nenhum de nós pensa realmente assim, mas é
útil ter em conta o que esperamos dos nossos relacionamentos. Realmente
não é fácil amar de forma que não tenhamos alguma exigência associada a
esse amor: «Gosto de ti desde que te portes como eu gosto, desde que não
faças nada que me aborreça.» - podemos colocar uma tremenda pressão uns
sobre os outros. Requer uma grande sabedoria ser capaz de amar uma
pessoa e mesmo assim \mbox{permitir-lhe} ser inteira e completamente ela
própria - deixá-la seguir o seu próprio caminho, a sua própria viagem.

Lembro-me que, quando decidi tornar-me monja, já tinha feito
anteriormente bastantes coisas que não agradaram aos meus pais - ou com
as quais não teriam ficado contentes, se tivessem sabido delas. Contudo,
tornar-me monja foi algo que fiz de forma bastante consciente; senti que
era algo que tinha de fazer. Pareceu-me um passo importante no sentido
de me tornar eu própria - libertar-me e, de certa forma, libertar
igualmente os meus pais. Num certo sentido foi bom não me ter apercebido
de como eles iriam ficar preocupados. Penso que teria sido muito
complicado avançar se tivesse sabido o quanto isso os iria afectar.
Evidentemente que eu não pretendia preocupá-los; não os queria
perturbar. O que me ajudou a ir para a frente com a decisão foi o
encorajamento de Ajahn Sumedho: «Se realmente quer ajudar os seus pais
(o que eu, evidentemente, queria) então torne-se monja e desenvolva um
pouco de sabedoria. Aí terá realmente qualquer coisa para lhes
oferecer.» Realmente, penso que isto era verdade. Penso que fui capaz de
lhes dar apoio, de formas que não me teriam sido possíveis se não
tivesse seguido este caminho. E tive a sorte de ter continuado a ter o
seu amor mesmo depois de me ter tornado monja.

Com frequência temos grupos de crianças em visitas de estudo a
Amaravati, o que é uma boa forma delas aprenderem sobre o Budismo. Por
vezes as crianças têm de fazer um trabalho sobre a visita de estudo e
uma vez, uma professora mandou-nos uma cópia do trabalho de uma das
crianças. Era de um menino de oito anos e achei notável o seu trabalho
escrito. Uma das coisas que comentou foi relativamente às seguintes
frases, que viu afixadas algures no mosteiro: «Se gostas muito de uma
coisa, deixa-a ir. Se for tua, voltará. Se não voltar é porque, de
qualquer maneira, nunca foi tua.» Acho esta reflexão bastante profunda.
Foi interessante que um menino tão novo a notasse e fixasse.

Existe uma palavra que recitamos nos nossos \emph{pujas} da manhã e da
noite, que se refere a uma das qualidades do Buddha que muito aprecio: é
a palavra «\emph{Sugato» -} «\emph{Gato}» quer dizer «ido» e a palavra
«\emph{su}» significa «bom» ou «bem». Entendo que esta palavra refere-se
ao Buddha como «aquele que vai bem»; alguém que é capaz de viver a vida
de forma plena e completa e, como é óbvio, de forma muito competente,
deixando o passado para trás. Acho isto útil porque noto que, por vezes,
posso ficar um pouco agarrada ao passado. Se cometo um erro, posso ficar
agarrada a ele e a minha cabeça continua a voltar atrás e a pensar sobre
o erro que cometi. É interessante verificar como a mente faz isso, como
tenta recriar, reescrever o passado, reescrever uma versão melhorada da
história, sem aquele erro terrível.

Se faço qualquer coisa bem, mais uma vez, a cabeça pode \mbox{agarrar-se} a
isso, enfatizando, alimentando o ego - «Sou uma pessoa fantástica!» Uma
vez ouvi um comentário num grupo de professores de Dhamma, que descreve
esta tendência para ficarmos agarrados, para nos identificarmos com algo
que tenhamos feito: «És tão bom quanto a tua última palestra.» Por isso,
até nos professores de Dhamma pode existir esta tendência para se
prenderem às coisas que fizeram, quer tenham sido bem ou mal feitas.
Quando reflectimos nesta qualidade do Buddha, \emph{Sugato},
apercebemo-nos que ele fez centenas, milhares, milhões de coisas muito
boas na sua vida - deu palestras de Dhama sublimes, confortou pessoas
que estavam em sofrimento e fez muitas outras coisas maravilhosas, mas
percebemos que ele era capaz de praticar estes actos e depois seguir com
a sua vida. Aquilo que nos indicava é que era iluminado. A minha
percepção é que esta era a sua forma de encorajar as pessoas a ter fé e
confiança nele - o tipo de fé que lhes permitiria aplicar os
ensinamentos que ele lhes deu, nas suas próprias vidas.

Deste modo, este epíteto, \emph{Sugato}, é algo que podemos contemplar.
Não se trata de não reflectir sobre as coisas que aconteceram, aprender
com os nossos erros e comemorar as coisas que fizemos bem - isso é
permitido, mas precisamos de ter muito cuidado, de forma a não criarmos
uma identificação pessoal à volta disto, quer como um «eu» desesperado e
terrível, quer como o «eu» maravilhoso, sublime e incrível - do tipo «eu
sou o melhor de todos.»

Com efeito, o Buddha disse que, pensarmos que somos melhores que os
outros é uma visão errada. Pensarmos que somos piores que todos os
outros, também é uma visão errada. Pensarmos que somos iguais aos outros
todos, é igualmente uma visão errada. Qualquer pensamento de nós
próprios como um «eu», ou~qualquer comparação com os outros enquanto
«eu», é uma visão errada.

Por isso, devemos estar conscientes desta tendência de nos criarmos a
nós próprios e de criarmos os outros. Evidentemente, não existe qualquer
dúvida que nós existimos de alguma forma. Temos estes corpos e mentes.
Mas se realmente gostamos e nos preocupamos uns com os outros, não nos
vamos criar nem fixar uns aos outros como uma personalidade. Através da
nossa prática de consciência podemos encontrar-nos uns com os outros
como pela primeira vez e permitir-nos mudar, em vez de nos agarrarmos a
uma imagem do que pensamos que os outros são, ou de como gostaríamos que
fossem. Uma das coisas que tenho apreciado verdadeiramente na comunidade
monástica, no Sangha, tem sido viver entre pessoas que têm pelo menos
tentado não se criar umas às outras.

Tentamos não guardar rancores, seja por não nos agarrarmos à imagem de
alguém como uma pessoa que comete erros, seja por não nos criarmos como
alguém que cometeu um erro. Se nos estamos a preparar para nos
encontrarmos com alguém, podemos observar como temos uma ideia sobre
essa pessoa e sobre a conversa que iremos ter, mas gostava de sugerir a
adopção de uma atitude mais de «nunca se sabe», para que a nossa
experiência de estar com essa pessoa seja mais directa, mais imediata,
em vez de estarmos com ela apenas através da ideia que temos dela.
Podemos vê-la como é realmente, no momento, e não através da imagem ou
da memória que temos dela.

Isto é algo que podemos ir investigando, à medida que cultivamos a
consciência. Somos capazes de largar as imagens que temos uns dos
outros? Somos capazes de permitir aos outros serem como são, em vez de
os fixarmos numa imagem estática, de acordo com a forma como gostaríamos
que fossem ou como pensamos que precisamos que eles sejam? Se
conseguirmos fazer isso a nossa vida torna-se muito mais rica e melhor.
É como se, em vez de vermos algo a preto e branco e a uma dimensão,
subitamente víssemos a cores e a três dimensões. Continuamos a ter
opiniões acerca das pessoas. Podemos na mesma antecipar com agrado estar
com alguém, ou recear estar com alguém. Podemos antecipar com agrado um
acontecimento futuro, ou recear um acontecimento futuro. Mas temos
igualmente a oportunidade de considerar que na realidade «não sabemos»,
realmente não fazemos ideia de como vai ser.

