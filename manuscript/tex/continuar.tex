\chapter{Continuar}

Temos praticado em conjunto, de forma diligente, em condições muito boas
e favoráveis, estabelecidas para nos permitirem chegar a um estado de
calma - pelo menos, certamente, a um estado de calma exterior. Alguns de
vós podem não se sentir ainda muito calmos por dentro mas só a vossa
vontade em estar aqui e de manter o nobre silêncio e em seguir a rotina,
tem sido um excelente apoio, uma grande ajuda. Apesar da maior parte de
vós não se conhecerem - talvez nunca se tenham encontrado anteriormente
-, partilham algo muito significativo: este interesse na prática, em
despertar, em libertar o coração do sofrimento. De forma que, durante
este tempo, tivemos oportunidade de praticar como uma comunidade,
ajudando-nos mutuamente neste caminho e tendo passado este tempo juntos
tranquilamente, existe um sentimento de camaradagem.

O Buddha disse que o mais importante para as pessoas, quando iniciam a
sua prática, é terem bons amigos. Precisamos de pessoas que nos possam
encorajar, apoiar e orientar. A primeira das grandes bênçãos listada no
\emph{Maha Mangala Sutta} é associarmo-nos a pessoas boas e evitarmos as
que nos levam na direcção errada, aquelas que nos encorajam a fazer
coisas que podem ser danosas para nós ou para os outros.

Quando cheguei à prática budista fiquei bastante surpreendida com a
ênfase colocada na virtude ou \emph{sila}, com a forma como vivemos as
nossas vidas. Já praticava meditação há algum tempo, inserida noutras
tradições e, de certa forma, tinha conseguido separar a meditação dos
outros aspectos da minha vida. A meditação era algo que eu fazia em
determinadas noites da semana e, noutras alturas, fazia outras coisas.
Mas o que mais me impressionou relativamente à forma como o Buddha
ensina foi o encorajamento para vermos o todo da nossa vida como
prática. Como vivemos, como falamos e como agimos - tudo faz parte da
nossa prática.

Quando olhamos para as nossas vidas, vemos que isto faz todo o sentido.
Começamos por reparar no efeito sobre a mente resultante de certas
formas de falar e de agir. Por exemplo, se estamos a coscuvilhar ou
dizemos alguma coisa cruel e indelicada, se falamos de forma descuidada,
isto tem um efeito negativo na mente. E o contrário também se verifica:
se estamos sempre a pensar de forma negativa ou crítica relativamente a
nós próprios ou aos outros, isto tende a reflectir-se no modo como
falamos. De igual modo, com acções, se fazemos coisas prestáveis, úteis
e gentis, isso tem um efeito positivo sobre a mente. Podemos ver isto
como uma espécie de terapia. Por vezes, quando estou a sentir-me um
pouco deprimida ou infeliz, faço, deliberadamente, algo gentil. Talvez
arrume alguma parte do mosteiro ou dê uma prenda a alguém, e sinto-me
sempre muito melhor depois disso. Enquanto se, de propósito, faço alguma
coisa um bocadinho má ou evito fazer certas coisas que deveria fazer,
embora possa retirar daí um certo tipo de satisfação, no meu coração não
me sinto muito bem.

De modo que pode ser muito útil reflectir sobre a importância de como
vivemos as nossas vidas. Por vezes, quando me sentia desencorajada com a
minha meditação, reflectia simplesmente sobre o facto de, apesar de
ainda não ser muito competente a acalmar a mente, conseguir fazer muitas
outras coisas boas - conseguia ser generosa, gentil e paciente. Então, à
medida que ia cultivando hábitos mais saudáveis na forma como vivia, fui
descobrindo um sentimento de totalidade, de perfeição no coração, um
sentimento de contentamento, um sentimento de alegria e isto, na
realidade, ajudou a minha meditação.

Existem vários ensinamentos nos quais o Buddha é directamente
questionado sobre os efeitos da generosidade, \emph{dana}, ou de se
seguir os preceitos, \emph{sila}, e todas as vezes ele refere este
sentimento de contentamento e de alegria que surge. O Buddha referia
igualmente que, quando temos cuidado e evitamos agir de forma danosa,
não temos de experienciar arrependimento ou remorso e, assim, o corpo
pode descontrair. Existe um sentimento de \mbox{bem-estar} físico, assim como
um sentimento de contentamento e de felicidade. Então a mente aquieta,
acalma-se de forma bastante natural. Estes são apenas os efeitos
naturais de viver de forma cuidadosa, gentil e generosa. E, à medida que
a mente se torna mais calma, mais concentrada, somos capazes de observar
mais de perto a maneira como a mente e o corpo funcionam, surgindo deste
modo sabedoria e discernimento.

Alguns de vocês podem reconhecer aqui a descrição do Caminho Óctuplo. A
imagem que é frequentemente usada para o simbolizar é a de uma roda com
oito raios. Tal como a roda, este Caminho Óctuplo gira num círculo. Aqui
começamos com o Discurso Correcto, a Acção Correcta e o Sustento
Correcto. Isto leva, muito naturalmente, ao Empenho Correcto, à
Consciência Correcta e à Concentração Correcta - a forma como acalmamos,
como tranquilizamos a mente e focamos a mente. Depois, à medida que a
mente se acalma, podemos ver de forma mais clara e dá-se o surgimento da
Compreensão Correcta e da Intenção Correcta, aspectos do discernimento e
da sabedoria. Chegamos à compreensão de como a vida funciona, da forma
como mente e corpo funcionam em conjunto. Chegamos à compreensão do
efeito do discurso e da acção da nossa mente. Assim, de forma bastante
natural, existe a propensão para falar e para agir de modo cuidadoso,
com habilidade. E assim sucessivamente, num círculo.

Este Caminho Óctuplo é aquilo a que o Buddha se referiu como o Caminho
do Meio, e as instruções que foram dadas é que este deve ser
desenvolvido. A pouco e pouco leva ao entendimento e à libertação de
todo o sofrimento ou luta. É interessante que no seu primeiro sermão, o
primeiro ensinamento do Buddha, ele aponte de forma tão directa para
este Caminho Óctuplo, este Caminho do Meio. Sinto-me sempre muito grata
que esteja expresso de forma tão clara, com instruções tão claras sobre
as coisas que posso fazer na minha própria vida. Também estou grata por
estar descrito como um caminho gradual, um treino gradual. Isto é algo
realmente importante para todos termos em consideração - o facto de não
acontecer imediatamente, não sermos instantaneamente libertos. Leva
tempo, esforço e muita paciência.

É por isso que necessitamos de bons amigos - para nos irem recordando,
para nos irem encorajando. Eles não o fazem necessariamente através de
palestras sobre o Dhamma ou mesmo dizendo seja o que for especificamente
acerca da forma como viver ou como praticar, mas mais através do seu
exemplo. Noto que aprendo muito mais a observar as outras pessoas - como
praticam, como fazem as coisas - do que com aquilo que elas dizem.
Assim, aprendo ao reparar quando as coisas correram bem, quando alguém
se comporta de forma cuidadosa e \mbox{sinto-me} encorajada e animada com isso.
Na nossa vida monástica temos uma hierarquia muito bem definida. As
pessoas que foram ordenadas há muito tempo são seniores e as pessoas que
estão há menos tempo são juniores - as pessoas que estão de visita não
estão incluídas na hierarquia. Contudo, noto que posso sentir-me
encorajada e inspirada seja por quem for, não apenas pelas pessoas
seniores, os professores, mas também pelas pessoas juniores e pelas
visitas. E isto não se restringe aos budistas - sinto contentamento e
alegria em estar seja com quem for que tenha assumido um compromisso
para com a Verdade, tal como é manifestada em qualquer tradição. Gosto
muito de pessoas boas!

De modo que o Caminho Óctuplo, a via que leva à cessação do sofrimento,
é a quarta das Quatro Nobres Verdades que o Buddha salientou no seu
primeiro sermão.

A Primeira Nobre Verdade é que a vida como ser humano é difícil. Algumas
pessoas acham que o Budismo é muito deprimente porque esta Primeira
Nobre Verdade diz simplesmente: «Existe sofrimento.» Contudo, quando
ouvi isto pela primeira vez pensei: «Felizmente!» Gostei mesmo desta
indicação precisa para algo e do reconhecimento de algo que, para mim,
era tão absolutamente óbvio na minha vida humana. Para mim era claro
que, mesmo que a vida por vezes pudesse ser bastante perfeita - às vezes
conseguia ter tudo exactamente da forma como queria -, iria alterar-se.
E, evidentemente, existiram bastantes alturas em que as coisas não
correram bem como eu queria. Em termos físicos existia desconforto -
dor, fome, sentir demasiado frio, sentir demasiado calor, sentir cansaço
ou ter demasiada energia. Com frequência não era «bem aquilo». Com a
mente era ainda pior. Havia alturas em que eu sentia irritação,
frustração, zanga, tristeza, depressão, confusão, falta de confiança,
inveja. É uma lista extensa, não é? E, por muito que eu não quisesse
estas coisas, elas lá estavam. Eu podia apreciar a vida quando elas não
estavam mas, depois, voltavam novamente. Assim, ouvir que o sofrimento é
um aspecto normal da vida e que alguém tão sábio como o Buddha estava a
frisar isto, constituiu um grande alívio.

% TODO: correct?

A instrução do Buddha para esta Nobre Verdade é que este sofrimento tem
de ser compreendido. Isto era interessante. Nunca me tinha ocorrido
tentar compreender o sofrimento que tinha experienciado na minha vida.
Estava muito mais interessada em ver-me livre dele tão depressa quanto
possível ou em distrair-me. Mas o Buddha disse claramente que o
sofrimento precisa de ser compreendido. Assim, para mim, a forma de
compreender uma coisa - e imagino que para vocês seja semelhante - é
\mbox{examinando-a}, tendo curiosidade acerca dela, interessando-me por ela.

Reflectindo sobre a Segunda Nobre Verdade, encontramos uma pista sobre
como fazer isto, uma vez que esta aponta para a origem do sofrimento -
existe uma razão pela qual sofremos, pela qual andamos em luta.
Basicamente, é por estarmos apegados a querer que as coisas sejam
diferentes daquilo que são. Em primeiro lugar, queremos ter experiências
deliciosas, lindas, agradáveis e maravilhosas. Em segundo lugar,
queremos evitar o que não nos agrada. E, finalmente, queremos existir -
como um «alguém» distinto e fascinante. Existe este sentido de si
próprio, que é condicionado e cuja existência é mantida com grande
empenho e que, no entanto, nos causa problemas tremendos.

Quando chegamos à Terceira Nobre Verdade verificamos que, se abdicarmos
destes três tipos de desejos, quando os abandonamos, experimentamos a
extinção do sofrimento. O sofrimento cessa. Experimentamos uma sensação
de paz e de à-vontade interior - por vezes é como a paz, a sensação de
calma depois de uma grande tempestade. Com frequência, quando estamos
realmente a lutar com algo, isto pode ser sentido como uma tempestade a
ocorrer no coração e, quando largamos, surge a calma.

Neste tipo de prática temos de ser muito honestos connosco próprios. Por
exemplo, podemos estar a lutar com algo e, apesar de fingirmos para
todas as outras pessoas que está tudo bem, dizendo: «Ah, não me importo.
Não tem qualquer importância!», connosco precisamos de reconhecer que,
na realidade, \emph{nos} importamos. Estamos descontentes. Estamos a
lutar com esta situação. Por isso pode parecer uma coisa estranha, uma
espécie de paradoxo na nossa prática, que, para libertarmos o coração
dessa luta seja preciso - antes de mais - reconhecermos que existe uma
luta. A primeira vez que me deparei com este ensinamento, lembro-me de
ter dito entusiasticamente a alguém: «Claro - toda a gente sofre.» E a
pessoa estava perfeitamente chocada - como se eu tivesse dito qualquer
coisa verdadeiramente terrível. Contudo, penso que muitas pessoas não se
dão conta que estão a lutar, que estão a sofrer. Podemos tornar-nos
peritos em fazer ficar «tudo bem» de uma forma muito superficial.

Estas verdades são intituladas «Nobres» porque elas fazem apelo a um
tipo de integridade, de honestidade, de vontade de olhar realmente para
todas as coisas das quais temos mais medo - onde nos sentimos mais
vulneráveis, confusos, assustados. Mas depois, tendo tido coragem para
olhar, podemos benefíciar do resultado de as reconhecer realmente e de
as deixar ir. E experimentamos uma espécie de liberdade e de leveza
notáveis. Tenho de admitir que me sinto sempre bastante grata em ter de
ser honesta apenas para comigo própria - não tenho de contar as minhas
lutas seja a quem for. Posso ter partilhado algumas das minhas lutas
convosco, agora neste retiro, como forma de vos encorajar a olhar com
maior profundidade e proximidade, mas posso \mbox{assegurar-vos} que, em geral,
na altura concreta de alguma dificuldade, nunca contaria fosse a quem
fosse! A maior parte das minhas lutas têm sido muito privadas, porque
com frequência é demasiado doloroso falar sobre elas a alguém. Nós mal
conseguimos contá-las a nós próprios. Mas quando o fazemos, vale bem a
pena o esforço.

Assim, a Primeira Nobre Verdade é que existe sofrimento e é necessário
compreender este sofrimento.

A Segunda Nobre Verdade é que existe uma origem do sofrimento, que é o
apego aos desejos, e é necessário abandonar estes desejos.

A Terceira Nobre Verdade é que o sofrimento cessa. Quando abandonamos ou
deixamos ir o desejo, reconhecemos a extinção do sofrimento; temos uma
experiência directa daquela paz depois da tempestade.

E a Quarta Nobre Verdade é que existe uma Via que leva à extinção do
sofrimento. Esta Via, este Caminho Óctuplo precisa de ser desenvolvido.

De forma que vos deixo estes pensamentos para a vossa reflexão e
encorajo-vos a continuar. Tendo posto os pés ao caminho, é mesmo muito
importante continuar. Não desistam. Por vezes as coisas vão parecer-vos
muito agradáveis e correr muito bem - vão sentir que estão mesmo a fazer
progressos. Outras vezes pode parecer que não vão tão bem e podem
sentir-se desencorajados. Mas todos os professores sábios ao longo dos
tempos têm dito simplesmente: «Continuem.» Ajahn Chah costumava falar
sobre a «prática da minhoca». As minhocas são criaturas bastante
simples. Não estão muito preocupadas se estão no início do caminho, ou
se estão quase no final, ou se estão a fazer grandes progressos. Tudo o
que fazem é continuar a trabalhar com aquilo que têm em frente delas, a
escavar a terra. Assim, encorajo-vos vivamente a continuar - a deixar
ir, a fazer as pazes com o que quer que esteja a acontecer e, por fim,
hão-de sair do outro lado.

