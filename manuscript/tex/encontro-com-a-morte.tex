\chapter{Encontro Com a Morte}

Fiquei impressionada com o afecto que me surgiu no coração quando vi um
dos participantes regressar ao retiro. Notei que outras pessoas também
se mostraram bastante contentes. Isto levou-me a pensar que, de certa
forma, apesar de não termos falado muito uns com os outros, aconteceu
algo entre nós durante estes dias - desenvolveu-se um certo sentido de
comunidade, como numa família. Por isso sentimo-nos um pouco tristes
quando alguém teve de sair mais cedo, teve de ir embora - ficámos
preocupados se estaria bem e, depois, quando voltou, ficámos contentes.

A minha percepção é que, quando praticamos desta forma em conjunto,
acontecem muitas coisas entre as pessoas, principalmente no silêncio -
vivemos em conjunto, levantamo-nos cedo, praticamos meditação, ouvimos
os ensinamentos e tratamos igualmente das coisas mais comuns que também
são necessárias, como cuidar do sítio onde estamos. Tenho notado durante
este tempo um sentimento de apreço por aqueles que se encarregam da
limpeza das casas de banho, que levam o lixo, que aspiram e, claro,
pelas pessoas da cozinha; para além disso, com todas as outras formas
através das quais cuidamos uns dos outros, simplesmente dando conta, de
forma tranquila, de coisas que nos pediram ou que nos oferecemos para
fazer. E, evidentemente, que isto inclui os que trabalham na informática
ou nos telefones, que fazem as listas e as compras - todas essas coisas.

É como se algo tivesse sido criado através da nossa vida em conjunto e,
apesar de ter sido durante um período de tempo muito curto, pode ser
bastante tangível. Evidentemente, em breve irá estar um grupo diferente
de pessoas aqui sentadas e a maior parte de nós irá estar noutro sítio
qualquer. Acho interessante olhar para o sentimento de contentamento e
de alegria quando se pensa neste tempo em conjunto e, depois, notar
também o ligeiro sentimento de tristeza, uma espécie de anseio no
coração, face ao pensamento da separação. Por vezes podemos pensar que
isto está errado - que não deveríamos sentir tristeza. Se fossemos
verdadeiros budistas seríamos completamente desapegados, não existiriam
quaisquer destas emoções desorganizadas que surgem quando não as
queremos e que achamos que não deveriam existir. Mas, tal como referi, o
que sinto é que a prática budista é bastante mais subtil do que
simplesmente não sentir as coisas. Quando falamos de desapego, de abrir
mão ou de equanimidade, a minha interpretação é que está a ser-nos dito
para não nos preocuparmos em demasia com aquilo que sentimos, para não
lutarmos contra isso. Se nos sentimos felizes e contentes, reparamos
nisso. Se sentimos tristeza e pesar, notamos isso. Se nos sentimos mesmo
irritados e zangados e confusos, também reparamos nisso, tornamo-nos
conscientes desse facto.

Quando estamos a crescer, somos geralmente ensinados em relação ao que é
certo sentir e àquilo que não devemos sentir. Por exemplo, na
Grã-Bretanha quando as crianças estão a crescer, há a tendência para
dizer aos rapazes que não devem chorar. Está bem que as raparigas chorem
mas não os rapazes e, certamente, não os homens. Dizem-nos também que
não devemos expressar irritação - pelo menos foi assim que fui educada.
E provavelmente fomos também condicionados a acreditar que muitas outras
coisas são também certas ou erradas, de forma que nos tornamos muito
hábeis em reprimir, em empurrar para baixo as coisas que não são
aceitáveis, mantendo-as sob um controlo apertado.

Contudo, virmos para um retiro constitui uma oportunidade para a mente e
o corpo se descontraírem, de forma que podemos experienciar todo o tipo
de coisas, mesmo num retiro de curta duração, como este. Penso que todos
necessitamos de encorajamento para não nos preocuparmos com estas
coisas, mas para simplesmente tomarmos consciência delas como num
processo muito natural e, depois, deixá-las ir. Algumas pessoas
chamam-lhe «lavagem ao cérebro», não no sentido habitual do termo mas
mais no sentido de nos permitirmos observar, notar, todas as coisas
reprimidas, à medida que surgem na consciência, de forma a podermos
deixá-las partir.

Aquilo que sinto é que, durante este retiro, as pessoas tiveram
oportunidade de experienciar bastante tranquilidade. Mas algumas pessoas
também descreveram terem tido muitos pensamentos e recordações de todos
os tipos. Por vezes as pessoas podem experienciar dor, que foi
reprimida, a qual nunca se permitiram realmente sentir na altura da
perda. Estas coisas podem parecer perturbadoras mas, na realidade, são
bastante boas. É muito melhor trazê-las à consciência, onde podemos
conhecê-las por aquilo que são, do que mantê-las sob a superfície da
consciência.

Algumas pessoas pensam que estão a enlouquecer quando vêm a um retiro:
surgem-lhes pensamentos malucos e imagens bizarras. Penso sempre nisto
como parte das limpezas gerais primaveris, não sendo, de todo,
preocupante. Algumas pessoas colocam questões relativamente a sensações
corporais pouco habituais, o coração a bater muito rápido ou sensações
de energias estranhas no corpo. Também são coisas que podem surgir
quando a mente se aquieta um pouco e, mais uma vez, não constituem uma
preocupação. A única preocupação seria se fossem tão agradáveis que nos
pudéssemos apegar a elas - e depois passássemos o resto das nossas vidas
a tentar voltar a senti-las. Isso seria lamentável. Na meditação podem
surgir todo o tipo de coisas: ter visões de luz, ouvir sons incomuns ou
ter sensações estranhas no corpo. Tudo isto só serve para observarmos,
para estarmos conscientes do seu surgimento e para reparar, igualmente,
quando cessam. Algumas pessoas experienciam estas coisas, outras não.

O desenvolvimento da consciência, esta forma de estabelecer um forte
sentido de presença, é um cultivo muito valioso. Algumas pessoas vão a
retiros apenas experienciar a mente tranquila e valorizam muito isso.
Mas essa sensação de tranquilidade está dependente de circunstâncias
muito particulares - longas horas de meditação, silêncio exterior e
todos os tipos de outras condições muito específicas que permitem e dão
o suporte ao sentimento de tranquilidade. Apesar de não ter nada contra
estas experiências, está claro para mim que, sob certos aspectos, são
bastante limitadas, uma vez que não podemos viver toda a nossa vida sob
estas condições. Temos de interagir com outras pessoas; mesmo ao
vivermos num mosteiro existe um elevado grau de interacção. A maior
parte das pessoas tem empregos. Muitas pessoas têm igualmente
responsabilidades familiares e muitas vivem em situações nas quais
aqueles que as rodeiam não estão particularmente interessados em
meditação. E, desta forma, precisamos de ter em consideração como a
prática pode ser mantida de forma significativa.

Como já referi, um dos meus \emph{suttas} preferidos é o Discurso das
Grandes Bênçãos. Este constitui a resposta do Buddha a um \emph{devata},
um ser celestial que, uma manhã cedo, o veio visitar. Diz-se que esta
era a altura do dia que o Buddha costumava instruir os \emph{devatas}.
Então, este ser particularmente radiante chegou e perguntou-lhe o que
proporcionava mais felicidade. O Buddha fez uma lista de bastantes
condições; por exemplo, conviver com pessoas boas, ter meios de sustento
adequados (algo que não envolva qualquer tipo de desonestidade ou de
dano), ter respeito pelos nossos pais e cuidar deles se necessário,
cuidar da nossa família, adoptar os Preceitos, evitar a intoxicação ou o
comportamento descuidado, cultivar a paciência, a gratidão e o
contentamento, e contemplar os ensinamentos. No último verso, que é
aquele de que eu realmente gosto, o Buddha diz: «Apesar de viver no
mundo, o coração não estremece. Está livre de mágoa, confusão e
carência.»

Acho tão inspiradora a existência da possibilidade de manter um
sentimento de estabilidade no mundo, onde temos de experienciar tantas
coisas diferentes, coisas boas e fantásticas mas também coisas muito
difíceis ou terríveis que podem abalar o coração, agitando-o ou
alterando-lhe o ritmo ou fazendo-o tremer. Chamamos a estas coisas as
Coisas Mundanas\footnote{\emph{Loka-dhamma}: Perda e ganho, honra e
  desonra, felicidade e tristeza, louvor e culpa.}. Mas vemos que,
através da consciência, temos efectivamente a possibilidade de nos
mantermos equilibrados mesmo quando sopra o vento mais forte. Não penso
que isto queira dizer que não sentimos as coisas - pelo menos espero que
não - mas mais no sentido de que podemos manter uma perspectiva
equilibrada relativamente ao que estamos a experienciar.

Há alguns anos morava connosco uma senhora de idade que se chamava
Nanda. Como era totalmente surda, tinha um grande aparelho auditivo mas,
mesmo assim, tínhamos de gritar muito alto para ela ouvir o que
dizíamos. A Nanda adorava o Dhamma e era muito dedicada ao Ajahn
Sumedho. A primeira vez que nos veio visitar, na casa das monjas, tinha
oitenta e três anos. A partir daí costumava visitar-nos frequentemente.
Teria vindo viver connosco de forma permanente, mas tinha um filho ao
qual estava muito apegada. A Nanda não gostava muito da mulher dele e
pensava que ela própria era a única pessoa que podia cuidar do filho de
forma adequada. Lá para o final da sua vida tornou-se uma espécie de
monja honorária. Levantou-se a questão se deveríamos rapar-lhe o cabelo,
mas ela tinha um cabelo branco encaracolado muito bonito e nunca
conseguiu cortá-lo. Penso que se preocupava com o que o filho iria
pensar. Assim, o seu grande acto de renúncia foi rapar as sobrancelhas.
Na tradição da Tailândia temos o hábito de rapar as sobrancelhas, para
além do cabelo.

Eu gostava muito da Nanda e costumava cuidar dela e ajudá-la. Por fim, a
senhora faleceu - foi a primeira morta que vi. Vestiram-na de
cor-de-rosa e colocaram-na num caixão - não parecia a pessoa que eu
conhecia. Houve uma grande cerimónia fúnebre e depois disso eu fiquei
completamente assoberbada com o pesar. O Ajahn Sumedho estava presente,
bem como bastantes outros monges e monjas e lá estava eu a chorar sem
parar - a sentir-me bastante envergonhada, porque estava certa que não
era correcto uma monja budista estar a chorar. Mas não o conseguia
evitar. Depois isso acabou.

Tivemos outra senhora de idade a viver connosco. Podia contar-vos ainda
mais histórias sobre ela. Também senti um grande pesar com a morte dela
mas, por essa altura, já tinha uma maior consciência. E, enquanto
anteriormente a ideia de perda e de pesar me parecia algo terrível, uma
experiência má e desagradável, quando tivemos o funeral desta senhora a
minha experiência emocional foi, na realidade, muito bonita. Existiram
momentos de grande tristeza, que surgiam com coisas pequenas, tal como
ver objectos que lhe tinham pertencido, mas houve também bastante
alegria. O funeral foi um momento de recordação feliz da vida desta
pessoa. Havia um grande sentimento de ternura e também de celebração.

Foi sepultada em Amaravati. Um dos monges tinha-lhe feito um caixão e as
monjas aprenderam a transportá-lo. Praticaram primeiro com sacas de
arroz, de forma a aprenderem a erguê-lo e depois a transportá-lo sem o
deixar cair. Na altura estavam a decorrer obras no mosteiro e a equipa
de trabalho tinha uma escavadora, de forma que fizeram o favor de nos
abrir a cova para o caixão e, todos os dias, as monjas praticavam
transportando o caixão e fazendo-o descer à campa. Não faço ideia do que
é que eles pensaram disto! Depois tivemos a cerimónia e todos
participaram e apesar de eu não ter transportado o caixão, ajudei a
descê-lo à campa. Todos atirámos flores e pequenas coisas que achámos
que ela teria apreciado. Depois pegámos em torrões de terra e atirámos.
O simples acto físico foi muito bom: o peso do caixão, mexer na terra e
fazer este ritual, esta cerimónia - ajudou muito. Foi sentido como uma
oferenda e uma forma de lidar com toda a emoção que ali estava. Por isso
foi sentido como uma expressão de amor, e de alegria e de tristeza,
todas estas coisas. Numa comunidade em que as pessoas estão conscientes,
estão presentes, há um sentimento grande de à-vontade relativamente às
emoções e às suas alterações - fosse o que fosse que estivesse a
passar-se, estava muito bem.

Com frequência penso que a nossa vida emocional é como o tempo. Na
Grã-Bretanha temos uma grande variedade de condições atmosféricas, todas
num dia: vento e chuva e trovoada e, depois, uma calmaria e um sol
radioso, tudo numa rápida sucessão. Foi assim no dia do funeral da Irmã
Uppala - na realidade, no funeral da Irmã Nanda foi semelhante.

Depois disto já não receio tanto o pesar e o luto, enquanto
anteriormente não conseguia imaginar como é que as pessoas lidavam com a
perda de alguém muito próximo. Evidentemente que não se trata de uma
experiência fácil, mas tenho descoberto que, quando consigo estar
presente, é suportável. De igual forma, é muito bom cuidar dos nossos
relacionamentos com os outros - é importante tentar resolver quaisquer
divergências, caso contrário pode ficar um sentimento de arrependimento
quando morre alguém que nos é próximo, um sentimento de termos perdido
uma oportunidade de corrigir a situação. Na realidade, ainda existem
formas de o fazer depois da pessoa ter partido mas, se possível, é muito
melhor fazê-lo antes de nos separarmos fisicamente.

O Buddha encoraja-nos vivamente a contemplar a nossa própria morte. Isto
pode ser bastante assustador, uma vez que a maior parte de nós não sabe
o que vai acontecer. Claro que algumas pessoas conseguem recordar
existências e mortes passadas mas, a maioria de nós, não tem essa
capacidade. Por isso, para nós a morte constitui algo que desconhecemos
totalmente, tudo aquilo que podemos ter a certeza é que, mais cedo ou
mais tarde, vai acontecer. Para mim uma contemplação útil é fazer as
pazes com a dúvida, com o sentimento de não saber: «Devo fazer isto?
Devo fazer aquilo? Não sei\ldots{}»

A maior parte de nós encontra-se viciada na certeza. Queremos saber.
Queremos saber onde estamos e o que vai acontecer a seguir. Quando
estamos num retiro, uma das coisas mais difíceis relativamente ao Nobre
Silêncio é que não podemos ter o tipo de conforto que estamos habituados
a ter uns dos outros. Alguém pode parecer não muito contente e podemos
partir do princípio que é por causa de alguma coisa que tenhamos feito,
mas quando não falamos, não há forma de nos certificarmos e podemos,
deste modo, sentir-nos muito inseguros. Assim, podemos reconhecer este
sentimento de incerteza, e como é desconfortável a sua presença nas
nossas vidas e, em vez de tentarmos o mais possível encontrar a certeza,
podemos refugiar-nos no Buddha, na consciência do momento presente.
Desta forma a vida torna-se bastante estimulante.

Imagino que o momento da morte pode ser verdadeiramente aterrador, se
não aprendemos a fazer as pazes com o sentimento de não saber o que vai
acontecer a seguir, se não fizemos verdadeiramente as pazes com isso.
Mas se o fizermos, se conseguirmos estar verdadeiramente à-vontade
apenas com esse sentimento de presença, penso que a morte pode ser
tremendamente excitante - dar um passo para além e entrar no
desconhecido. Houve uma vez que ia morrendo e compreendi que poderia ter
morrido sem sequer me aperceber disso. Do meu ponto de vista, na altura,
tudo me pareceu um absoluto não-acontecimento. Mas, evidentemente, cada
caso deve ser diferente, por isso na realidade não sei\ldots{}

Durante este retiro tivemos uma oportunidade de ter em consideração
aquilo que é realmente importante na vida - de rever as nossas
prioridades e de reflectir sobre o que queremos para o resto das nossas
vidas. Evidentemente, não espero que tenhamos desenvolvido um plano
detalhado, mas talvez tenhamos começado a apreciar a possibilidade de
estarmos completamente presentes, realmente presentes para experienciar
a vida inteira e completamente, em vez de vivermos numa recordação do
passado ou num sonho sobre o futuro. Claro que podemos reflectir sobre
os acontecimentos do passado. Podemos comemorar as coisas boas que nos
aconteceram ou das quais fizemos parte, e podemos permitir que esses
pensamentos nos tragam um sentimento de alegria ao coração, e também
podemos aprender com os erros do passado. E é claro que existem alturas
em que é necessário fazermos algum tipo de planos para o futuro, por
exemplo como é que vamos para casa amanhã, depois do retiro. São coisas
práticas sobre as quais temos de pensar e depois disso, podemos
concentrar toda a nossa consciência no aqui e agora.

Assim, se alguém me pergunta qual a melhor forma de se preparar para a
morte, a minha resposta será: cultivar a consciência, a presença, viver
de forma cuidadosa e responsável e assegurar-se que disfruta a vida.
Assegurar-se que faz as coisas que são importantes para si, de forma
que, quando chegar ao fim, não pense: «Ai, porque é que não fiz aquilo?»

