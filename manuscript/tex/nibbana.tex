\chapter{Nibbana}

Quando praticamos em conjunto, nos retiros, durante um certo período de
tempo, tudo começa a aquietar. Sente-se uma maior acalmia na sala, de
forma que conseguimos ver o efeito deste tipo de estrutura de retiro na
mente. As coisas começam a acalmar naturalmente quando deixamos de nos
estimular, de receber os mais diversos estímulos. Evidentemente que os
processos internos da mente persistem, os pensamentos, as memórias que
surgem, etc, mas pelo menos quando nos mantemos quietos e não vemos
televisão nem andamos de um lado para o outro numa grande cidade, a
mente é menos agitada. As coisas começam a acalmar.

Alguém estava a colocar uma questão relativamente ao \emph{Nibbana}, por
isso pensei falar um pouco sobre este assunto. Trata-se daquilo que
todos nos esforçamos por alcançar, por isso penso que é bom termos uma
ideia para onde vamos. É algo do qual o Buddha falou muito, sob
diferentes formas, e usou muitas analogias diferentes para se referir à
experiência do \emph{Nibbana}. A própria palavra é interessante, porque
trata-se de uma palavra que era usada de forma bastante normal no tempo
do Buddha - quer simplesmente dizer «arrefecido». Era uma palavra
utilizada quando por exemplo se cozia arroz: quando o arroz acaba de ser
cozido está muito quente e não o conseguimos comer - temos de esperar
até que arrefeça um pouco para o comer. Assim, \emph{Nibbana} era a
palavra usada para descrever o arroz quando está suficientemente frio
para ser comido - uma sensação de arrefecimento sereno, uma sensação de
à-vontade.

Há uma frase que o Buddha usa com bastante frequência nos \emph{suttas,}
quando fala aos seus discípulos: «Por não compreendermos quatro coisas,
tanto eu como vocês tivemos de despender muitas vidas nesta caminhada
difícil dos renascimentos, tendo permanecido no \emph{samsara}» - que é
o reino de errância contínua no qual tentamos sempre encontrar algum
conforto, alguma paz, alguma satisfação. Podemos sentir-nos confortáveis
durante algum tempo mas, depois, ficamos novamente desconfortáveis e
temos de nos mover; sempre em movimento, sempre errantes, à procura de
algo confortável e tranquilo. Ou então, algumas pessoas podem procurar
alguma forma de se manterem estimuladas e estão sempre à procura de
experiências mais excitantes, fantásticas e emocionantes. Por vezes,
podemos encontrar isto mesmo em círculos espirituais, pessoas que vão a
este e àquele retiro, e a este e àquele professor, e procuram o mosteiro
perfeito. Ajahn Chah costumava provocar os seus discípulos, dizendo que
era inútil procurar o mosteiro perfeito: pura e simplesmente não existe.
Penso que se trata de uma tendência que todos podemos reconhecer nas
nossas próprias mentes: estamos sempre em movimento, a tentar encontrar
algo melhor, a tentar vermo-nos livres das coisas das quais não
gostamos, a tentar manter as que nos agradam.

O próprio Buddha teve experiências muito semelhantes na sua vida. Como
sabem, ele cresceu num ambiente muito privilegiado, sendo filho de um
poderoso monarca, e toda a sua juventude decorreu num grande conforto.
Sempre que queria qualquer coisa, era-lhe dada. Na realidade, o seu pai
queria que ele ficasse completamente satisfeito com a sua vida enquanto
príncipe, porque quando ele nasceu tinha havido uma profecia, de acordo
com a qual aquela criança tornar-se-ia um soberano admirável - um
soberano muito poderoso e bem-sucedido - ou um Buddha, um ser iluminado.
E, como muitos pais, o seu pai queria que o Buddha seguisse o que
\emph{ele} tinha feito, ser um monarca, não queria que o Buddha seguisse
uma via religiosa. De forma que fez tudo o que podia para o manter
satisfeito, para o manter feliz. Mas, um dia, o jovem príncipe encontrou
aqueles que designamos como os Quatro Mensageiros Celestiais.

Os mensageiros budistas não são como os anjos cristãos, que são
frequentemente representados como seres lindíssimos com asas fofinhas. O
Mensageiro Celeste era uma pessoa de idade que mal conseguia andar, não
ouvia nem via muito bem, não tinha muitos dentes, com a pele toda
engelhada e com o cabelo grisalho e a cair - uma figura muito patética.
O segundo era alguém muito doente, num estado imundo e repugnante, com
diarreia e a vomitar, claramente num estado de grande desconforto, com o
corpo coberto de chagas. O Terceiro Mensageiro Celeste era o corpo de um
morto no cortejo de um funeral. Se já viajaram pela Índia e viram
cortejos fúnebres, sabem que podem ser bastante espectaculares,
acompanhados pelo bater de tambores, à medida que o corpo é levado pelas
ruas. Quando o jovem príncipe viu estes três primeiros Mensageiros
Celestes ficou profundamente perturbado. Perguntou ao seu companheiro o
que eram e este disse-lhe: «Bem, aquele é uma pessoa de idade, aquele
uma pessoa doente e aquele o corpo de um morto, uma pessoa morta.» O
príncipe perguntou: «Aquilo é o que acontece a todas as pessoas?» O
condutor respondeu: «Sim, todas as pessoas envelhecem e, mais tarde ou
mais cedo, morrem. E, evidentemente, as pessoas adoecem.» De modo que o
príncipe ficou muito perturbado com isto, porque mostrava o vazio e a
futilidade de todos os valores com os quais tinha crescido.

O Quarto Mensageiro Celeste era uma pessoa religiosa tranquilamente
sentada, serena, debaixo de uma árvore, obviamente com perfeita
consciência do que se estava a passar e, sem contudo ser perturbada por
isso. Isto foi o que levou o Príncipe a deixar o palácio: parecia-lhe
claro que a única opção era seguir a via religiosa, de forma a dar
sentido àquilo que tinha visto. Naquele tempo havia muitas pessoas que
procuravam a via religiosa e que se envolviam nas práticas ascéticas
mais austeras - jejuavam durante dias e dias, não se deitavam para
dormir, meditavam durante muitas horas e, de uma forma geral, privavam o
corpo de tudo o que era agradável, num esforço para vencer os desejos
dos sentidos. De modo que o Príncipe tentou estas práticas, tendo sido
excelente em todas elas. Mas, por fim, ficou demasiado fraco para poder
prosseguir. Percebeu igualmente que não se tinha aproximado mais daquilo
que procurava - tinha apenas ficado extremamente fraco e magro.
Felizmente para nós, surgiu alguém que lhe ofereceu algum arroz de
leite, que ele aceitou e, gradualmente, foi recuperando um pouco de
força.

Então o Buddha lembrou-se de uma experiência da sua infância, quando o
seu pai o tinha levado a um festival de lavoura\footnote{No original
  \emph{ploughing festival} - cerimónia tradicional de lavoura efetuada
  para assegurar boas colheitas.}. Como os adultos estavam todos
envolvidos no festival, deixaram a criança só, debaixo de uma árvore.
Ele devia ter à volta de sete ou oito anos: sentou-se de pernas cruzadas
e ficou consciente da sua respiração, de uma forma muito fácil e
natural. Esteve ali sentado durante muitas horas a disfrutar esta
experiência. Podemos designar isto como \emph{jhana}, um estado de
absorção. Finalmente, os adultos regressaram e, segundo a história,
encontraram-no sentado à sombra desta árvore. Uma vez que ele tinha ali
permanecido muitas horas, o sol tinha-se movido mas, segundo a lenda, a
sombra manteve-se sobre ele.

À medida que o Príncipe recordou esta experiência, pensou que este era
um tipo de prazer inofensivo, que não se encontrava relacionado com um
desejo forte. E compreendeu que talvez não devesse recear estas
experiências agradáveis, e que talvez estas até o pudessem ajudar na sua
busca. De forma que voltou à meditação com o seu corpo um pouco mais
nutrido e saudável e, por fim, alcançou o entendimento perfeito - o
estado de \emph{Nibbana}. Continuou a disfrutar este agradável estado
durante várias semanas e, por fim, começou a partilhar com outros o seu
entendimento, tendo-o formulado como As Quatro Nobres Verdades. Estas
são quatro coisas que o Buddha disse ser necessário compreender para nos
libertarmos do \emph{samsara}, de forma a sair do reino da errância
contínua. Estas Nobre Verdades são o \emph{dukkha} (sofrimento), a sua
causa, a sua extinção e o caminho para a sua extinção.

De modo que, tal como Buddha ensinou, experienciamos a extinção do
sofrimento quando abdicamos do desejo: quer seja desejo de prazeres
sensoriais, desejo de ser ou desejo de não ser. Mas apenas podemos
renunciar a estes desejos quando tomamos consciência de como é
insatisfatório realizá-los, quando compreendemos plenamente que eles
nunca nos poderão realmente satisfazer. Isto acontece porque todas as
condições - mesmo as mais agradáveis - são impermanentes, ou
\emph{annica}. Para além disso, são todas \emph{anatta -} não existe uma
pessoalidade inerente a qualquer condição de mente ou de corpo. Penso
que, provavelmente, cada um de nós tem tido um pouco de entendimento
relativamente a isso. Talvez seja isso que nos trouxe aqui, um
sentimento que deve existir algo mais na vida do que conseguir ter
aquilo que se quer!

É difícil reconhecer plenamente que não existe pessoalidade em qualquer
condição da mente ou do corpo, porque temos o forte hábito de nos
identificarmos com o corpo, a mente, as nossas qualidades, os nossos
pontos fortes, as nossas fraquezas, o nosso carácter, a nossa
personalidade. Deste modo, não estamos assim tão preparados para
abdicarmos dessa identificação. Afinal de contas despendemos uma grande
quantidade de energia na criação e na manutenção do nosso «eu». Mas o
Buddha encoraja-nos a desafiar esse sentido de identidade - isto
torna-se mais relevante e interessante, quando compreendemos realmente o
sofrimento que causamos a nós próprios com esta identificação habitual.

É interessante que o Buddha, em vez de referir-se a si próprio como
«eu», descreve-se como o \emph{Tathagata}, o «deste forma ido», ou pode
ser «desta forma vindo», não é claro, mas, basicamente, apenas um ser
que existe. O Buddha tinha certamente uma personalidade, tinha
seguramente muitos talentos, era um professor extremamente dotado e
muito compassivo, e também tinha um bom sentido de humor - mas estas não
eram qualidades com as quais se identificasse, fosse de que forma fosse.
Era apenas um sentimento de «ser», de conhecer as coisas tal como elas
são, uma consciência - não como nós, que podemos estar tão preocupados
com a nossa aparência, ou com aquilo que fazemos bem, ou com os erros
que cometemos. Os nossos sucessos e insucessos têm para nós uma grande
importância.

Também nos identificamos com a nossa nacionalidade, o nosso género, o
nosso nível etário, as nossas filiações políticas, a nossa família, a
nossa profissão. Existem tantos rótulos que colamos a nós próprios,
formas através das quais nos descrevemos e que têm uma realidade
convencional, mas nós também temos de ter consciência da realidade
fundamental da nossa existência, que não se identifica com qualquer um
destes aspectos. Quando estamos plenamente presentes, não pensamos em
nós próprios como sendo professores, monges ou monjas, médicos ou
enfermeiros, ou seja o que for. Existe apenas a experiência da presença.
Mas se alguém surge e nos desafia, nos critica, ou critica alguém que
nos é próximo ou de quem gostamos, podemos ficar muito zangados e
transtornados. Isto acontece porque nos identificámos, porque estamos,
de alguma forma, apegados.

Não estou a dizer isto para fazer alguém sentir-se mal relativamente a
estar apegado. Frequentemente, acontece pura e simplesmente que estamos
apegados, mas o mais importante é notar que isto causa sofrimento, de
modo que vos encorajo a interessarem-\linebreak-se por isto, a tentarem notar o
grau em que estão apegados a variadas coisas e como isso pode ser
doloroso, porque só aí é que vão sentir-se verdadeiramente motivados
para tentarem abdicar disso.

O Buddha usou muitas analogias ao referir-se a diferentes aspectos da
prática. Existe um \emph{sutta} muito bonito no qual ele fala dos seus
ensinamentos e da forma de praticar comparando-os ao oceano. Diz que,
tal como o oceano se inclina de forma muito suave, do mesmo modo a
prática deve ser bastante gradual. A pouco e pouco começamos a notar que
já não nos aborrecemos tanto com coisas que costumavam perturbar-nos.
Disse também que tal como o oceano tem um sabor, o do sal, também esta
forma de prática tem um sabor, o da liberdade. De modo que vos encorajo
a continuar a investigar, a continuar a observar as formas através das
quais criam uma noção de vocês próprios - todas as ideias que têm acerca
de quem são e daquilo que têm de fazer, e se a vossa prática é boa ou
não, ou se algum dia vai ser boa. Podem colocar-se a questão: isto é
verdade? É mesmo assim? Tenho mesmo de acreditar nestas coisas?

Desta maneira, gradualmente o sentido de pessoalidade é dissolvido. À
medida que vemos a forma como o criamos, começamos a ter consciência da
possibilidade de não o continuar a criar e, ao invés, movermo-nos pela
vida sem nos agarrarmos, sem apego. Esta é a experiência do
\emph{Nibbana}, na qual deixa de haver qualquer atracção e apego seja ao
que for, na qual a nossa vida torna-se simplesmente uma resposta
proveniente de um sentimento de compaixão e de uma visão límpida, de uma
compreensão límpida.

De modo que ainda existe muito trabalho para todos nós fazermos, mas
espero ter sido capaz de dar algum sentido àquilo que estamos a tentar
fazer. Ajahn Chah costumava perguntar às pessoas quando chegavam ao seu
mosteiro: «Veio aqui para morrer?» Penso que, provavelmente, ficavam um
pouco chocadas com esta pergunta, mas aquilo que ele estava a frisar era
a prática de morrer antes de morrermos - permitir que o sentido do ego e
o sentido da pessoalidade se desmoronem, de forma a vivermos no Dhamma,
ao invés de num anseio egoísta. Costumava ficar bastante espantada
quando as pessoas falavam em não ter qualquer desejo. Lembro-me de
perguntar a Ajahn Chah e a Ajahn Sumedho (estavam lá os dois) se era
errado desejar muito o Dhamma. Eles perguntaram se isso me causava
sofrimento, ao que respondi: «Não.» Assim, podemos ver que existe um
desejo que leva a mais sofrimento, mas existe igualmente um desejo que
leva ao fim do sofrimento. Penso que todos gostaríamos de acabar com o
nosso sofrimento, não é verdade? Então, este é um desejo e está tudo
bem. A expressão em Pali para isto é \emph{dhammachanda}, entusiasmo
pelo Dhamma, amor ao Dhamma.

