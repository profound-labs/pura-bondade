\chapter{Introdução}

Esta pequena colectânea de ensinamentos é maioritariamente composta por
palestras oferecidas num retiro de uma semana, realizado na República
Checa em 2009. A palestra final foi dada no Retiro de Final de Ano, em
2011/12, também na República Checa.

Apresentei os ensinamentos em inglês e estes foram traduzidos para
checo, para benefício da maioria dos participantes, com escassos ou
nulos conhecimentos de inglês. Isto exigiu uma precisão e economia de
palavras que descobri ser uma disciplina muito útil a partir do momento
em que me adaptei a um ritmo mais lento. Os intervalos para a tradução
também permitiram uma pausa meditativa - não constituíram, como alguns
imaginaram, momentos para pensar sobre o que dizer a seguir! A minha
ideia inicial era preparar um pequeno livro com as transcrições em
inglês e em checo, lado a lado. Posteriormente tornou-se claro que a
melhor abordagem seria preparar e editar primeiro uma versão em inglês.
É esta que o leitor tem agora nas suas mãos (traduzida desta vez para
Português). Uma equipa de tradutores checos aguardam para posteriormente
preparar a versão checa para publicação.

Um objectivo deste livro é fornecer uma introdução para quem vai pela
primeira vez a um retiro com um professor da nossa tradição Theravada.
São apresentados os ensinamentos básicos, a par de algumas estruturas de
apoio para a prática (os Refúgios e os Preceitos), retornando a eles -
por vezes duas ou mesmo três vezes - de forma a que o leitor, ou
participante no retiro, seja constantemente lembrado dos temas
essenciais da prática. À medida que avança nestas páginas, não se
surpreenda se encontrar várias vezes tópicos como as Quatro Nobres
Verdades, impermanência, ``insatisfação'' e «não-eu», ou bondade e
curiosidade como antídotos para os cinco obstáculos.

Aqueles que já têm vários anos de prática budista contemplativa podem
igualmente encontrar alento e apoio ao regressar a estes temas básicos.
À medida que os ensinamentos são aplicados nas nossas vidas, trabalham o
coração e, gradualmente, transformam a nossa visão do mundo, pelo que,
de cada vez que os ouvimos, algo de novo é revelado.

Há mais de 2500 anos o Buddha exortou os seus discípulos «\ldots{} ide e
deambulai para o bem e a felicidade de muitos, por compaixão para com o
mundo, para o benefício, bem e felicidade dos deuses e da humanidade.» É
neste mesmo espírito que estes ensinamentos são, mais uma vez,
apresentados.

\bigskip

{\raggedleft
Irmã Candasiri\\
Rocana Vihara\\
Abril de 2012
\par}


