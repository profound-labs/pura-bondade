\chapter{Curiosidade e Compreensão}

Ao chegarmos ao final do primeiro dia de prática, é interessante reparar
nos efeitos deste tipo de esforço. Cada um pode reparar quão diferente
se sente em relação a como se sentia ontem por esta hora, quando aqui
chegaram. Vejo que estão mais ambientados e, talvez, um pouco cansados.
Isto é normal, uma vez que também vejo que todos se têm aplicado muito
na prática da meditação. Por vezes a meditação requer mesmo que nos
apliquemos bastante. Esta noite gostava de vos falar sobre diferentes
tipos de compreensão. A palavra ``compreensão'' pode ser usada para
\emph{qualquer} tipo de constatação - um daqueles momentos de «ahhhh»
quando, de repente, vemos as coisas de forma diferente, quando temos uma
compreensão diferente. A compreensão pode ser acerca de coisas bastante
vulgares. Como exemplo, gostava de vos falar de uma compreensão que me
ocorreu, recentemente, acerca de algo pequeno e prático. Relaciona-se
com a casa, ou \emph{kuti}, onde vivo. Trata-se de um \emph{kuti} muito
simpático, que foi construído há cerca de catorze anos por uma das
monjas; está ainda em muito boas condições. Contudo, há alguns anos
comecei a notar que existiam algumas criaturas a viver nas paredes e no
tecto. Não tinha a certeza do que eram, mas desde que tivemos um
problema com ratazanas no mosteiro, aqueles que organizam o trabalho
partiram do princípio que se tratava de uma ratazana e resolveram pôr
rede de arame à volta da base do meu \emph{kuti}, em todos os sítios por
onde uma ratazana pudesse entrar. Fizeram um excelente trabalho\ldots{}
mas, claramente, as criaturas ainda viviam lá felizes e contentes.

Foi para mim um quebra-cabeças tentar descobrir como é que entravam e
saíam.Nos último meses tornaram-se muito mais activas, de uma forma
bastante desagradável. Ao princípio costumava achar que elas eram
bastante amorosas, mas agora acho-as verdadeiramente intrusivas, por
isso estava decidida a descobrir como é que elas entravam e saíam, de
forma a poder fazer qualquer coisa para as impedir. Pensei que talvez
estivessem a escavar túneis por baixo do \emph{kuti}, de forma que
sempre que encontrava um buraco que parecesse uma possibilidade nesse
sentido colocava uma pedra sobre ele. Mas elas continuavam lá. Então
comecei a ouvir atentamente. Parecia que estavam a saltar para o
telhado, vindas das árvores mais próximas. Pedi a alguém que cortasse os
ramos das árvores, de forma que não pudessem ter esse acesso ao telhado.
Mas isso também não resultou, elas continuavam lá. Então ouvi dizer que
existem ratazanas que conseguem trepar pelas paredes dos edifícios.
Pensei que estas deviam ser dessa espécie, por isso despendi muito tempo
a ver se existiam quaisquer buracos na parte de baixo do telhado. Mas
trata-se de um \emph{kuti} muito bem construído, e não consegui
encontrar quaisquer buracos. Então comecei a olhar para as telhas para
ver se as ratazanas poderiam entrar por baixo delas, mas também não
consegui ver aí quaisquer buracos.

Eventualmente, em desespero, subi por um escadote de forma a olhar
devidamente para o telhado. Então surgiu o momento da compreensão:
\emph{Ahhhh!} Vi três buracos grandes na base da chaminé. Assim, quando
achei que as ratazanas tinham saído, fechámos esses buracos. Mas a
história não acaba aqui, porque nessa noite houve muita agitação e o som
de qualquer coisa a roer a parede ou o telhado. Bem, isso confirmava que
era pelos buracos que elas entravam \emph{e} saiam. De forma que ainda
não tínhamos conseguido resolver o que fazer. Destapámos os buracos e eu
estou apenas à espera que saiam - e a entoar um cântico intitulado o
cântico de proteção das cobras:

\begin{quote}
  \itshape
  Gosto dos animais sem pés\\
  Gosto dos animais com dois pés\\
  Gosto dos animais com quatro pés\\
  Gosto dos animais com muitos pés\\
  Mas por favor vão-se embora.
\end{quote}

Espero que isto vos dê uma ideia de um momento de compreensão, de quando
estamos perplexos relativamente a algo, investigamos verdadeiramente e
\emph{depois} vemos. Com a nossa prática aplica-se o mesmo princípio. Os
ensinamentos do Buddha existem para nos encorajar e mostrar para onde
dirigir as nossas vidas. Podemos estudar muitas escrituras de forma a
conseguir uma compreensão intelectual, mas sinto que todos aqui estão
interessados em algo mais nutritivo para o coração, de outra forma
estariam todos na universidade a estudar Budismo!

Quando ouvi pela primeira vez ensinamentos budistas, na realidade não me
apercebi que se tratava de Budismo; pareceram-me apenas senso comum e
fiquei muito entusiasmada relativamente à possibilidade de
\emph{descobrir} realmente a verdade por mim própria. Assim, ao virmos
fazer este retiro, temos uma oportunidade de contemplar os ensinamentos
em relação à nossa própria experiência. Aqui, a teoria e a prática
encontram-se e, quando nos aplicamos, a compreensão pode surgir.

Uma coisa que tenho notado relativamente às compreensões é que estas
muitas vezes surgem de forma bastante inesperada. Podemos sentar-nos ou
fazer meditação a andar durante horas e nada acontece. Mas depois,
quando estamos a fazer qualquer coisa muito normal, como lavar os
dentes, vestir-nos ou sair para andar um pouco - de repente, há
compreensão. Vemos o que um determinado ensinamento quer dizer -
torna-se real para nós.

O Ajahn Chah contou uma história sobre uma compreensão que certa vez
teve: ele queria fazer um hábito para si próprio; ora, os nossos hábitos
têm um modelo especial - são cozidos de uma forma particular, bastante
complicada. Então Ajahn Chah passou o dia inteiro a pensar nisso e,
finalmente, conseguiu perceber através do seu raciocínio, como fazer o
hábito. Com esta experiência Ajahn Chah constatou que, quando estamos
realmente interessados em qualquer coisa, de forma natural aplicamos o
nosso raciocínio e, a seu tempo, surge a compreensão. É por isso que dou
sempre ênfase e encorajo uma atitude de curiosidade - sermos curiosos
relativamente à nossa experiência. Se realmente nos interessarmos e
aplicarmos o nosso esforço a investigar a vida, começamos a compreender.

Podemos ter compreensões relativamente a muitas coisas, algumas das
quais são úteis do ponto de vista da libertação, enquanto outras não o
são de todo. Penso que todos nós, aqui presentes, estamos interessados
no tipo de compreensões que nos trarão liberdade ao coração, que nos
libertem o coração. Queremos compreender a razão pela qual sofremos.
Porque é que existe sofrimento? Queremos compreender como libertar o
coração do sofrimento. Bem, viemos ter ao lugar certo, porque o Buddha
disse: «Ensino sobre o sofrimento e sobre o fim do sofrimento.»

Falei ontem das Quatro Nobres Verdades. Para cada uma das verdades o
Buddha apontou três aspectos. O primeiro aspecto da Primeira Nobre
Verdade é que existe sofrimento. O segundo aspecto é que o sofrimento
precisa de ser compreendido, depois vem a realização libertadora de que
o sofrimento \emph{foi} compreendido. Se queremos compreender algo, é
necessário que estejamos dispostos a examiná-lo. Contudo, esta não é a
resposta comum ao sofrimento. Em geral, quando existe sofrimento,
queremos ver-nos livres dele tão depressa quanto possível, não queremos
examiná-lo. Mas se seguirmos a prática como o Buddha ensinou, torna-se
claro para nós que, se queremos libertar o coração do sofrimento será
necessário examiná-lo - precisamos de o compreender. Podemos passar toda
a nossa vida a tentar distrairmo-nos do sofrimento, mas isso não faz com
que estejamos mais perto da libertação do coração.

Fiz o meu primeiro retiro com o Ajahn Sumedho há cerca de trinta anos,
como leiga. Na minha entrevista ele perguntou como é que eu estava.
Disse-lhe que gostava muito do retiro, estava a gostar muito da
estrutura do retiro\ldots{} e então desatei a chorar e dei por mim a
dizer: «Mas tenho este orgulho. Não consigo ver-me livre dos pensamentos
de orgulho!» Eu pensava realmente que não devia ter estes pensamentos e
queria desesperadamente ver-me livre deles. Ajahn Sumedho e o monge que
o acompanhava ficaram muito silenciosos. Passados uns momentos Ajahn
Sumedho disse-me: «Não é o \emph{orgulho} que é o problema. É \emph{não
o querer}.» Este foi um momento importante para mim. Vi claramente a
distinção entre aquilo que estava a acontecer na minha mente e a minha
reacção a isso. Pude ver que existia uma distinção clara entre a
condição desagradável e o sofrimento. Comecei a ver que, mesmo com
situações desagradáveis da mente ou do corpo, podemos, na realidade,
estar bastante serenos. Tornou-se claro que o sofrimento não era causado
pela própria condição mas pela luta no sentido de evitar essa condição.

Querer que as coisas sejam distintas daquilo que são, envolvermo-nos no
desejo de que as coisas sejam diferentes, leva-nos ao stress, à luta, ao
sofrimento. Mas se pudermos reconhecer a condição desagradável e
simplesmente admitirmos que existe um desejo natural de querer evitá-la,
podemos alcançar uma condição de paz real. Isto também funciona com
condições agradáveis: temos a tendência de nos agarrarmos a elas, não
queremos que elas se alterem mas, é claro que também elas mudam. Desta
forma, também podemos sofrer com condições agradáveis, se não
compreendermos e aceitarmos realmente o facto da mudança.

Durante este retiro temos a oportunidade de investigar a nossa
experiência, para sermos verdadeiramente curiosos e interessarmo-nos
pelas nossas respostas às diferentes situações que surgem. Por exemplo,
alguns de vocês não estão muito bem, estão constipados ou têm a febre
dos fenos, e portanto podem notar o tipo de negatividade e de aversão
que surge em resposta às sensações físicas desagradáveis - e,
evidentemente, que o mesmo se aplica a condições mentais desagradáveis.

Um outro aspecto importante da compreensão relaciona-se com
\emph{annica}, impermanência, mudança. Uma vez estava a fazer uma
caminhada com um grupo de leigos. Estávamos a caminhar há cerca de uma
semana numa zona de campo muito bonita, no norte de Inglaterra. Um dia,
estava junto a uma cascata a observar o movimento e os padrões da água
que caía. Ocorreu-me um poema, e constatei que na realidade este
reflectia a vida:

\begin{quote}
  \emph{A cair}\\
  \emph{A cair}\\
  \emph{Posso apanhar?}\\
  \emph{Posso apanhar?}\\
  \emph{Não\ldots{}}\\
  Não podemos!
\end{quote}

Nas nossas vidas por vezes experienciamos coisas lindíssimas. Temos
também relacionamentos especiais, nos quais sentimos uma afinidade
maravilhosa e um grande à-vontade com determinadas pessoas. Pode
acontecer o desejo de nos agarrarmos a essas experiências. Mas
precisamos compreender que a vida envolve uma espécie de
implacabilidade. É \emph{annica} - simplesmente continua a fluir, quer
queiramos ou não.

O \emph{Dhammacakkappavattana Sutta} é o primeiro sermão do Buddha
depois do seu despertar. Nele o Buddha fala acerca das Quatro Nobres
Verdades e dos seus três aspectos - este \emph{sutta} tem uma quantidade
imensa de ensinamentos. Diz-se que depois do Buddha ter apresentado este
ensinamento, o Venerável Kondañña, um dos cinco ascetas que se
encontravam a ouvir, \emph{compreendeu}. O sutta prossegue dizendo que
ele compreendeu que: «Tudo o que tem a natureza de surgir tem a natureza
de cessar.» Gosto muito desta simplicidade. Somos também informados que,
quando o Venerável Kondañña compreendeu isto, todos os \emph{devas},
todos os anjos de todos os reinos celestes se regozijaram. Os
\emph{devas} terrenos gritaram para o nível seguinte: «O Venerável
Kondañña compreendeu! Ele percebeu!» e o regozijo ascendeu através de
todos os diferentes níveis dos reinos dos \emph{devas} e o mundo de dez
mil sistemas agitou-se, abanou e estremeceu. Houve uma enorme tempestade
e um esplendor sem limites invadiu todo o universo - tudo devido à
compreensão do Venerável Kondañña de que tudo o que tem a natureza de
surgir tem a natureza de cessar. Para alguns de vocês esta compreensão
pode parecer pouco importante, mas foi obviamente muitíssimo
significativa. É algo que podemos contemplar ao longo destes dias, de
forma a que compreendamos que tudo - estes corpos, estas mentes, os
estados de espírito, os estados do corpo - está em constante mudança.

Uma das coisas que aprecio quando regresso a este centro de retiros é
ter a oportunidade de ver as mudanças: as árvores que se plantaram a si
próprias no rio, a crescer, o edifício a degradar-se, as modificações
nas pessoas que encontro ano após ano e ver a minha própria cara no
espelho, ver como se altera. Para pessoas que não contemplam o
\emph{Dhamma}, ver a cara a mudar com a idade pode ser um mau prenúncio.
Mas eu gosto bastante.

Para concluir, as compreensões libertadoras relacionadas com as Quatro
Nobres Verdades são compreensões relativas às três características de
todas as condições: tudo é impermanente (\emph{anicca}), tudo é
inerentemente insatisfatório (\emph{dukkha}) e não existe qualquer
pessoalidade permanente seja no que for (\emph{anatta}). Existe uma
frase que aprecio muito nos cânticos da manhã: «Para a total compreensão
disto, durante a sua vida, o Abençoado instruiu frequentemente os seus
discípulos neste sentido.» Entoamos este cântico praticamente todos os
dias no mosteiro. Acho este cântico diário muito útil pois este
ensinamento pode levar algum tempo até ser realmente compreendido.

A minha expectativa para estes dias de retiro é que, na meditação, todos
experienciem alguma calma, alguns estados de espírito agradáveis, e
espero igualmente que todos tenham uma pequena compreensão relativamente
à natureza do sofrimento e à possibilidade de o coração se libertar
dele.

