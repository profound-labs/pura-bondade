\chapter{Os Cinco Obstáculos}

Estamos a praticar em conjunto há alguns dias. Até agora, tem sido dada
maior enfâse à meditação focada na serenidade e também falei um pouco
sobre as compreensões. Muitos de nós, principalmente os que já praticam
há algum tempo, têm já uma noção clara daquilo que é necessário. Mas,
apesar da prática ser simples, nem sempre é fácil. Isto acontece devido
aos hábitos da ignorância e da ilusão (que nos têm condicionado desde
que nascemos) serem muito fortes. Queremos muitas coisas e não queremos
muitas outras. Em geral temos alguma confusão relativamente ao que é bom
para nós e, por isso, fico sempre grata ao Buddha por ter formulado de
modo tão claro os ensinamentos e a forma de praticar.

Muitos de vocês têm noção de que no Budismo existem muitas listas. Isto
aconteceu porque no tempo do Buddha os ensinamentos não eram escritos,
não havia qualquer registo e as pessoas não tomavam notas como hoje em
dia fazemos. Elas reuniam-se em redor do Buddha, escutavam os seus
ensinamentos e memorizavam-nos. Aparentemente as pessoas desses tempos
tinham memórias prodigiosas, podiam lembrar-se de um número formidável
de coisas. Penso que os registos e os computadores, e esse tipo de
coisas que temos hoje em dia, nos tornaram um pouco preguiçosos -- os
nossos cérebros não têm a mesma capacidade, não temos tanta memória.
Mais facilmente confiamos na memória do nosso computador! Mas, naqueles
tempos, os discípulos do Buddha lembravam-se dos ensinamentos e, quando
estavam em viagem ou longe do Buddha, repetiam-nos para as pessoas que
se reuniam para ouvi-los. Os ensinamentos foram transmitidos desta forma
durante várias centenas de anos, antes de terem sido compilados e
escritos. Isto parece completamente espantoso quando vemos as
colectâneas das escrituras do Buddha pois contêm milhares de páginas e
uma quantidade vastíssima de informação.

Uma das listas, que considero muito útil na minha prática, é a que se
refere aos Cinco Obstáculos. Esta lista explica, de forma muito clara, a
razão pela qual, por vezes, apesar de sabermos o que \emph{devíamos} de
fazer, não o conseguimos fazer. Estes obstáculos são impedimentos.
Dificultam-nos a meditação. Mas quando temos consciência dos obstáculos,
podemos ver o caminho através deles e então, ao invés de obstáculos,
eles podem levar-nos a um estado de maior discernimento e de compreensão
mais profunda. Assim, na realidade, eles podem por vezes ser muito
úteis, uma vez que tenhamos aprendido a reconhecê-los. Primeiro vou
enumerá-los e depois falarei um pouco sobre cada um deles. Penso que
reconhecerão alguns, senão todos, na vossa própria prática.

% TODO: spacing

\begin{itemize}
  \item O primeiro obstáculo na lista é o desejo sensorial, a ânsia sensorial.
  \item O segundo obstáculo é a má vontade, a negatividade, a aversão.
  \item O terceiro obstáculo é a sonolência e a lentidão (preguiça e torpor).
  \item O quarto obstáculo é a inquietação e a agitação da mente - \emph{uddhacca-kukkucca} em Pali.
  \item O quinto obstáculo é a dúvida.
\end{itemize}

Tal como referi, quando os reconhecemos claramente, deixam de ser
obstáculos, uma vez que nessa altura existem maneiras muito claras de
trabalhar cada um deles. Só são obstáculos quando ainda não os tornámos
conscientes, quando ainda não estamos realmente conscientes do que está
a acontecer. Metemo-nos em sarilhos quando os seguimos ou somos levados
a reagir a eles. É aí que eles nos prendem e evitam que vejamos as
coisas claramente.

Uma das coisas mais difíceis no que diz respeito aos obstáculos, é que
temos tendência a complicá-los, por lhes termos aversão.

Podemos transtornar-nos por sermos gananciosos. Sentimos que não
deveríamos ser gananciosos e, por isso, detestamo-nos. Detestamos ser
irritáveis e rabugentos e negativos. Podemos sentir-nos completamente
desesperados com a nossa molenguice e sonolência, podemos sentir-nos
completamente frustrados e perturbados com a nossa mente inquieta e
agitada e podemos debater-nos seriamente com a dúvida, não a querendo,
não gostando dela. Assim, complicamos ainda mais estas condições com a
nossa aversão e resposta negativa a elas. Mas quando simplesmente
reconhecemos a ganância, a negatividade, a preguiça, a inquietação e a
dúvida -- quando podemos reconhecê-las por aquilo que são, em vez de as
encararmos como um problema, algo que não deveríamos ter ou uma qualquer
falha básica no nosso carácter -- tornam-se um quebra-cabeças ou um
desafio interessante, uma parte do jogo da vida. Podem até tornar-se
divertidas.

Evidentemente, nem sempre é fácil vê-las sob este prisma e por vezes
necessitamos de um amigo próximo que nos ajude a ver as coisas de forma
diferente. Essa é uma das maiores bênçãos de viver num mosteiro. A maior
parte dos monges e das monjas, particularmente os que já lá se encontram
há muito tempo, já perderam um pouco o idealismo relativamente a eles
próprios ou a qualquer pessoa. Já não têm expectativas no sentido de
serem sempre perfeitos. São capazes de reconhecer e aceitar os
obstáculos e compreendem que estes são uma parte natural da condição
humana. Não algo a ser detestado ou temido, mas sim algo a ser
compreendido e trabalhado de forma positiva.

Em Chithurst, nos primeiros tempos, todos trabalhávamos arduamente. A
casa encontrava-se praticamente em ruínas e tivemos de a reconstruir. O
jardim estava uma selva. Éramos quatro monjas noviças e costumávamos
cozinhar para todos. Por vezes não havia muito para cozinhar: teríamos
feijões e arroz e urtigas e, talvez, uma cebola. Era muito divertido,
mas a nossa meditação não era grande coisa(!) (Estaríamos) todas
sentadas a cabecear e sonolentas! Lembro-me do Ajahn Sucitto, um dos
monges seniores, elogiar-nos e encorajar-nos. Ele dizia que devíamos
comemorar o facto de, pelo menos, conseguirmos manter-nos sentadas sem
bater com as cabeças no chão. Quando vivemos numa comunidade de
praticantes, é possível rir (acerca) destas coisas, não as levando
demasiado a sério. Mas quando não temos este tipo de apoio, podemos
sentir-nos muito desesperados quando experienciamos estados de ganância,
estados de negatividade ou estados de embotamento. Só o simples facto de
conseguirmos notar o que se está a passar, pode requerer, uma tremenda
quantidade de compaixão e de ternura para connosco.

Creio que a razão pela qual muitos de nós procura a meditação prende-se
com o facto de querermos tornar-nos pessoas melhores. Mas, claro que,
como provavelmente se aperceberam, assim que começamos a meditar tomamos
consciência de todas as coisas menos boas acerca de nós próprios. Quando
era leiga costumava pensar que era uma pessoa bastante pacífica. As
pessoas diziam: «Ela é tão calma, tão tranquila.» Mas quando me tornei
monja descobri uma raiva imensa; por vezes apetecia-me bater nas pessoas
ou desejava que não existissem! Por isso poderiam pensar que, talvez, eu
não devesse ter começado a meditar, mas discordo com veemência. Penso
que é muito bom meditar. Podemos encarar a meditação como um processo de
descontracção interna. Somos educados para sermos bem comportados e
bons, de modo que aprendemos a reprimir ou ignorar as nossas qualidades
desagradáveis. Depois, à medida que começamos a meditar, descontraímos
e, assim, estas coisas começam a vir à superfície. No Budismo falamos
sobre purificar o coração e fazemos isto ao permitir que todas estas
coisas, que não parecem assim tão puras nem encantadoras, venham ao de
cima. Então podemos largá-las. Por vezes, descobrimos que posteriormente
temos muito mais energia, pois já não estamos a despender muito esforço
a tentar manter tudo sob controlo.

Os Cinco Preceitos são uma protecção. Quando os adoptamos abstemo-nos de
matar, de roubar, de ter uma conduta sexual danosa, de ter um discurso
incorrecto e de consumir substâncias intoxicantes. São como cinco bons
amigos que nos recordam o cuidado a ter com as energias do desejo, da
luxúria, da negatividade, do ódio e da aversão. Assim, quando estas
energias surgem, podemos simplesmente reconhecê-las e tratá-las com
cuidado e respeito, uma vez que podem causar bastantes danos. E, à
medida que as vamos compreendendo e percebendo como lidar com elas,
deixamos de ter medo delas. Quando, através da nossa investigação, se
torna claro que são \emph{anicca}, \emph{dukkha}, \emph{anatta} - mudam,
são insatisfatórias e não são quem somos - deixamos de precisar de nos
identificar com elas e, então, de certa forma, libertamo-nos delas.
Mesmo que elas possam lá estar, já não estamos presos ao desejo de as
seguir, de agir de acordo com elas. Temos escolha.

Nas escrituras, o Buddha dá uma série de ensinamentos sobre os
obstáculos e as diferentes formas de trabalhar com eles, utilizando
comparações muito úteis, para nos dar uma ideia do que são e de como nos
afectam.

A má vontade ou negatividade, disse o Buddha, é como uma doença. Durante
os retiros é muito importante estar atento às diferentes formas de
negatividade, particularmente a negatividade dirigida a nós próprios.
Podemos ser muito duros connosco próprios, muito intolerantes. Parecemos
ter muita destreza para ver os nossos defeitos, para reconhecer as
coisas que fizemos mal mas, por vezes, temos uma verdadeira dificuldade
em celebrar as coisas que fazemos bem. Isto é algo ao qual temos de
estar atentos e no qual temos de reparar.

Estes obstáculos podem ser muito grandes, muito radicais, mas podem
igualmente ser muito subtis. Por vezes, é mais difícil lidar com as
formas mais subtis de negatividade. A raiva é muito óbvia, mas a pequena
irritação ou a rabugice podem, à partida, ser mais difíceis de
reconhecer. Uma das coisas que tenho reparado na minha própria prática
no que respeita à rabugice ou à irritabilidade é que, com frequência,
acho que não as devia sentir. Contudo, aprendi ao longo dos anos a
simplesmente reconhecer quando me sinto rabugenta. Por vezes, digo mesmo
às pessoas: «Não falem comigo hoje, estou a sentir-me rabugenta.» E, por
vezes, se alguém que conheço muito bem fala comigo de coisas que me
perturbam, então digo-lhe: «Não estou com disposição para falar disso
agora.»

Assim, é realmente útil reconhecer quando existe negatividade ou aversão
na mente pois de outra forma esta pode facilmente intensificar-se e
afectar os outros. Lembro-me uma vez, de estar a conversar com alguém
sobre um assunto no qual não estava particularmente interessada, mas fiz
um comentário qualquer e a pessoa discordou de mim. Então,
automaticamente, eu discordei dela e depois ela discordou de mim! E dei
por mim a ficar cada vez mais zangada e desagradada. Eventualmente vi o
que estava a fazer e descontraí-me. Reconheci aquele sentimento de ter
de vencer uma discussão com alguém e pensei: «Na realidade não tenho de
fazer isto.»

Fiquei muito impressionada com um grupo de pessoas que trabalham na
Palestina, para ultrapassar os conflitos entre os palestinianos e os
israelitas. O seu lema é «A Paz é o mais importante.» Apesar disto me
fazer todo o sentido, podia ver que a prática destas pessoas era muito
exigente. Estabelecer a paz significa abdicar de ou renunciar à posição
de ter razão. É um verdadeiro sacrifício do ego, do sentido de si
próprio. E, devido à forma como fomos condicionados, o sentido de si
próprio é \emph{muito} importante para nós. Apesar de ser apenas uma
ilusão, fazemos tudo o que está ao nosso alcance para nos agarrarmos a
ele, até nos apercebermos de todos os problemas que causa, tanto a nós
como aos outros.

Assim, a negatividade ou a má vontade é como uma doença.

O Buddha disse que o desejo sensual é como estar em dívida, sentir que
se tivéssemos aquilo que queremos seríamos completos e felizes. Mas
\emph{nunca} nos sentimos completos, queremos sempre cada vez mais e
melhor - quer seja uma relação maravilhosa, algum tipo de comida
deliciosa, um computador novo, um telemóvel, seja o que for. O Buddha
encoraja-nos a contemplar o desejo sensual e a ter cuidadosamente em
conta o facto de que tudo está em constante mudança. Uma prática que
tenho considerado útil, é contemplar a comida deliciosa \emph{antes} de
a comer e, seguidamente, considerar o que acontece \emph{depois} de a
comer. Torna-se menos desejável, não é verdade? Assim, como monásticos,
somos encorajados a ingerir a comida apenas como um medicamento, como
algo que nutre e sustém o corpo. Certamente que precisamos de comer e
não existe nada de errado relativamente a isso, mas as pessoas podem
criar todo o tipo de complicações à volta da comida. Lembro-me de uma
altura em que tinha receio de ser gananciosa, de forma que pensei que
devia comer simplesmente pouco, não me apercebendo que, na realidade, me
estava a matar à fome. Por vezes, pode ser difícil separar o desejo por
comida da sua necessidade como um apoio para o bem-estar do corpo.

A preguiça e o torpor são comparados com a estadia numa prisão: é como
se estivéssemos numa cela pequena e escura de uma prisão, sem conseguir
ver nada para fora, presos num sítio exíguo. Um dos antídotos sugeridos
é contemplar a luz. Assim, quando me sinto muito sonolenta, ajuda-me
olhar fixamente para uma vela, abrir os olhos e olhar para a sua chama.
Outros antídotos que o Buddha recomendou incluíam puxar os lóbulos das
orelhas, lavar a cara com água fria e, se tudo isto falha, sentarmo-nos
na beira de um precipício - nunca me atrevi a fazer isso!

A inquietação e a agitação são comparadas a sermos como um criado a quem
é dito para ir para aqui e para ali, fazer isto, fazer aquilo, andar de
um lado para o outro. Por vezes, quando estamos preocupados com alguma
coisa a mente pode tornar-se muito activa. Algumas pessoas podem viver
muito ocupadas; eu tenho tendência para andar sempre muito ocupada,
gosto de fazer coisas. Mas, quando estamos ocupados de uma forma
inconsciente, cria-se uma energia muito desagradável para os que nos
rodeiam. Com isto não quero dizer que não devemos fazer coisas, mas é
muito melhor fazê-las com base numa calma e tranquilidade internas, em
vez de corrermos de um lado para o outro «como uma barata tonta», como
se costuma dizer. Por isso tentamos evitar fazer isso estando tranquilos
e presentes. Quando experienciamos inquietação e agitação internas, o
Buddha encoraja-nos a concentrarmo-nos em algo que acalme e estabilize.
Assim, o meu antídoto favorito para a inquietação e agitação é andar no
meu caminho de meditação, e estar pura e simplesmente consciente dos pés
a tocar no chão, pois os pés estão muito longe da cabeça; os pés não
pensam.

O último obstáculo é a dúvida, e esta é comparável a estar perdido no
deserto, sem um mapa, sem saber para que lado ir. Tenho verificado que
me podem acontecer duas coisas quando não estou consciente da dúvida.
Uma é uma espécie de paralisia - não faço nada. A outra é tentar fazer
coisas, umas a seguir às outras, e sem ter a certeza - tentar encontrar
uma resposta, de forma a ver-me livre da sensação de dúvida. Mas hoje em
dia, de uma forma geral, não me importo com a dúvida. A dúvida pode
levar-nos para um ponto muito calmo no coração; quando não sabemos mesmo
e podemos \emph{permitir-nos} não saber, a tranquilidade acontece.
Quando temos uma decisão muito importante a tomar, se nos podemos
permitir ir para esse ponto de não saber, é quase como nos permitirmos a
que a «mente sábia» surja. Ao longo dos anos aprendi a confiar muito
mais na voz do coração do que na voz da cabeça.

Assim, aqui vai uma recapitulação dos cinco obstáculos que, tal como
referi, não precisam de ser obstáculos, se estivermos conscientes deles
- são obstáculos apenas quando reagimos a eles.

% TODO: spacing

\begin{itemize}
  \item Em primeiro lugar o desejo sensual, que é como sentirmo-nos em dívida.
  \item Em segundo lugar a aversão ou negatividade, que é como uma doença.
  \item Em terceiro lugar a preguiça e o torpor, que é como estar na prisão, numa cela pequena.
  \item Em quarto lugar, a inquietação e a agitação, que é como ser um criado a quem ordenam ir para aqui e para ali.
  \item E em quinto lugar, a dúvida, que é como estar perdido num deserto, sem mapa.
\end{itemize}

Espero que isto tenha sido útil, e que estejam aptos a reconhecer
quaisquer destes obstáculos quando surgem na vossa prática, em vez de
lhes adicionarem outros obstáculos ao dizer «Não devia sentir isto.
Pratico há tantos anos e já devia ter ultrapassado a minha irritação.
Sou um caso perdido pois estou sempre a deixar-me dormir. Como é que
consigo que a minha mente pare de pensar?!? A ganância é um problema
terrível para mim.» Em vez de fazerem isto, pura e simplesmente
reconheçam: «Ah, isto é apenas torpor.», «Ah, isto é apenas
preocupação.», «Que interessante, porque é que me sinto tão zangado?»
Desta forma tornamo-nos amigos de nós próprios, começamos a conhecer-nos
e a interessar-nos por estas condições à medida que surgem, em vez de as
encarar como problemas terríveis que nos transformam num caso perdido.

Temos uma figura no Budismo chamada Mara, que nos quer ver
desencorajados, a pensar que somos casos perdidos e não quer que
despertemos. É um pouco como o Satanás do Cristianismo. A resposta de
Jesus a Satanás foi: «Vade retro (para trás) Satanás.» De forma
semelhante, a resposta do Buddha a Mara foi: «Conheço-te, Mara.» O pobre
Mara sentiu-se sempre muito desencorajado quando o Buddha o reconheceu.

