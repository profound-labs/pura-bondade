\chapter{Intenção Correcta}

\emph{Pergunta:} Tendo notado algo e tendo-o aceite, fica-se por aí?
Ou há mais alguma coisa que devamos fazer?

\emph{Resposta:} Esta é uma questão que tem muito a ver com a nossa
prática diária de meditação bem como com a prática do nosso dia-a-dia, a
forma como vivemos no mundo.

O Buddha falou sobre o Caminho Óctuplo e os seus oito factores:
Pensamento Correcto, Intenção Correcta, Acção Correcta, Discurso
Correcto, Modo de Vida Correcto, Empenho Correcto, Consciência Correcta
e Concentração Correcta. Talvez o factor mais relevante para esta
questão seja a Intenção Correcta. Se a intenção vier de um desejo por
prazeres sensoriais, da aversão ou da crueldade, é melhor não a seguir,
ao passo que as intenções de generosidade, de bondade e de compaixão,
são intenções sãs e é bom segui-las. A intenção de libertar o coração é,
evidentemente, a melhor de todas - a de compreender e libertar o coração
do sofrimento.

Primeiro vou falar um pouco sobre a Intenção Correcta no mundo e depois
falarei da Intenção Correcta na meditação. A~forma como vivemos as
nossas vidas tem um efeito na nossa meditação.

O facto de toda a nossa estrutura monástica se encontrar assente na
generosidade, sempre me despertou o interesse. Os monges e as monjas não
poderiam existir sem a generosidade das pessoas leigas. O Buddha
reconheceu que uma das maiores causas da miséria humana é o egoísmo, o
desejo de obter mais e mais coisas e o medo de não ter o suficiente, ou
de se perder aquilo que se tem. Assim que começamos a praticar
generosidade, a partilhar o que temos com os outros, verificamos que, na
realidade, há sempre mais do que o suficiente. Descobrimos igualmente
que o coração é nutrido de forma bastante especial, quando começamos a
ser generosos.

Em algumas tradições Budistas existe o hábito de colocar de lado um
pouco da comida que se recebe para os fantasmas que têm fome. Quando
estava a viver na floresta em Chithurst, certa vez decidi experimentar
esta prática e partilhar um pouco da minha comida, cada dia, com os
animais da floresta. Foi muito interessante notar a sensação de prazer
que resultou de pôr de lado um pouco de todos os tipos de comida, mesmo
daqueles que eu mais gostava. Mesmo quando havia menos do que eu
normalmente gostaria, era muito bom partilhar aquilo que tinha.

Há uma história do tempo do Buddha, sobre de uma senhora muito sábia,
chamada Visakha. Uma vez ela decidiu convidar o Buddha e os monges para
uma refeição em sua casa. Quando a refeição estava pronta ela mandou a
criada chamar os monges. Quando a criada ia para o parque onde estavam
os monges, caiu uma trovoada terrível e os monges tinham despido os
hábitos e estavam à chuva, a apreciar a água que caía sobre eles. A
criada chegou ao parque e viu todas aquelas pessoas despidas. Voltou
para Visakha e disse-lhe: «Desculpe. Não consegui encontrar os monges.
Estava lá apenas um grupo de ascetas nus.» Visakha era muito inteligente
e compreendeu o que devia ter acontecido. Por essa altura já a chuva
tinha parado, de forma que disse: «A chuva parou. Vai agora e
convida-os.»

Eventualmente os monges chegaram a sua casa e depois de ter oferecido a
comida, Visakha disse para o Buddha: «Gostava de pedir oito favores
especiais.» Então o Buddha perguntou: «Quais são?» Visakha disse:
«Gostava de dar panos de banho para todo o Sangha, porque a nudez é
imprópria.» E continuou a listar as outras coisas que queria oferecer,
por exemplo, comida especial para os que estavam doentes e para os que
cuidavam deles, comida para os que estavam a partir de viagem e para os
que acabavam de chegar à cidade, e um fornecimento constante de papas de
arroz, que é uma comida medicinal muito agradável. Ao todo eram oito
pedidos.

O Buddha, então, disse: «Isso é maravilhoso, mas que proveito é que
retira disso para si?» Visakha disse: «Vêm aqui muitos monges e monjas e
se eu souber que eles foram bem-sucedidos nas suas práticas isso
dar-me-á satisfação. Sentir-me-ei feliz. Estando feliz o meu corpo
ficará descontraído. Sentirei tranquilidade e bem-estar. Estando num
estado de tranquilidade \mbox{bem-estar}, a minha mente irá ficar concentrada.
Isso irá sustentar o aparecimento dos factores da libertação. São estes
os proveitos que eu vejo.»

Acho esta história interessante, uma vez que mostra que, a partir de
algo muito simples como uma acção generosa, há uma progressão natural,
que leva à libertação perfeita. Um processo semelhante é descrito quando
o venerável Ananda interroga o Buddha acerca dos benefícios de
\emph{sīla} (moralidade). A resposta do Buddha foi que, devido a
\emph{sīla}, existirá uma ausência de remorso, o que dará lugar a um
sentimento de contentamento e, por aí fora, na mesma progressão, passo a
passo.

Quando reflectimos nas nossas vidas, compreendemos que fizemos coisas
menos boas. É interessante notar a forma como a mente fica perturbada
quando fazemos algo danoso ou egoísta. Inversamente, quando conseguimos
evitar fazer algo desse género e ou quando fazemos algo bom, sentimo-nos
bem com isso - sentimo-nos contentes. Existe uma ligação muito óbvia
entre a forma como vivemos as nossas vidas e como nos sentimos
relativamente a nós próprios, e o modo como estamos na meditação.

Somos igualmente encorajados a abrir mão. Se cometemos um erro, somos
encorajados a aprender com ele e depois seguir em frente. Temos de tomar
cuidado para evitar cair na armadilha de nos sentirmos culpados e
terríveis, e de pensarmos que somos pessoas horríveis. Isso não traz
qualquer benefício. É muito mais importante simplesmente reconhecer que
cometemos um erro, ter em conta o que aconteceu e, depois, tentar ser
mais consciente no futuro. É desta forma que aprendemos.

Um dos equívocos relativamente ao Budismo é a ideia de que é muito
passivo. Parte-se do princípio que, como Budistas, apenas nos sentamos e
aceitamos tudo; que não criamos problemas seja com o que for e que
fazemos as pazes com tudo. É~compreensível que se possa ter essa
impressão mas, de acordo com aquilo que entendo dos ensinamentos do
Buddha, e tendo em conta o exemplo da própria vida do Buddha, vejo que o
que se pretende é algo diferente. O Buddha não indicou que nunca devemos
reagir, mas sim que as nossas acções e o nosso discurso devem resultar
da Compreensão Correcta e da Intenção Correcta, de uma intenção sã. É
aqui que se torna importante, antes de mais, aceitar a realidade e fazer
as pazes com ela.

Sabemos de tantas coisas no mundo que nos fazem ficar zangados e
perturbados. De tão zangados que estamos, podemos querer partir para
mudar o mundo, mas não sei se recomendaria isso! Contudo, quando levamos
tudo em consideração, de uma forma profunda, podemos chegar a um estado
de aceitação e de compreensão de que estas coisas perturbadoras fazem
parte das dificuldades humanas. As pessoas fazem coisas danosas por
ignorância e também vemos o mal que fazem a si próprias. Quando temos em
conta as coisas desta forma, a nossa reacção é mais compassiva. Talvez
então, possamos reflectir sobre como podemos ajudar a mudar esta
situação horrível. «O que posso fazer para ajudar?» é uma reacção
diferente de: «Vou lá e vou resolver aquilo! Isto é terrível, estas
coisas não deviam acontecer!» Uma reacção muito melhor seria: «Isto é
terrível, como é que eu posso ajudar as pessoas a compreender e a agir
de forma diferente?» Assim podemos fazer muitas coisas, dependendo das
nossas capacidades e dos nossos interesses específicos. Há muitas formas
de sermos úteis, por isso é bom considerarmos cuidadosamente, dentro do
que tomamos conhecimento, se existe algo que possamos fazer.

Pode ser que, em algumas situações, não exista nada de concreto a fazer.
Se visitarmos alguém que está doente ou perto da morte, ou alguém que
está a chorar uma perda, pode ser que não consigamos fazer nada para
aliviar a sua dor ou mágoa. Pode acontecer que, nestas situações, apenas
ser capaz de aceitar e fazer as pazes com esse sentimento de impotência,
e estar presente com a pessoa com um coração em paz, seja a atitude mais
prestável e compassiva. Por vezes isso é muito mais útil do que andar de
um lado para o outro a fazer uma série de coisas práticas, que podem
fazer com que \emph{sintamos} que estamos a fazer algo útil mas que, na
realidade, não têm um efeito positivo, se partem meramente de um estado
de agitação interna. À medida que praticamos a nossa meditação e nos
tornamos mais aptos a fazer as pazes com as emoções e os sentimentos
difíceis, podemos contribuir com algo muito mais subtil, muito mais
profundo.

Assim, por vezes existem coisas práticas que podemos fazer e outras
vezes não, mas só a nossa presença já pode ser um apoio. Foi
interessante reparar nas diferentes formas pelas quais as pessoas se
relacionaram comigo quando estive doente. Nessas alturas temos tendência
a estar mais sensíveis, de forma que eu conseguia sentir se alguém não
se sentia à-vontade relativamente à minha doença. As pessoas mais
prestáveis eram aquelas que conseguiam apenas «estar», sem dizer ou
fazer necessariamente grande coisa.

No outro dia estava a falar com alguém sobre o Discurso Correcto,
particularmente em relação a algo que vemos que não é correcto, que
sentimos que não é como devia ser - por exemplo, se sentimos que alguém
se está a aproveitar de nós ou a fazer qualquer coisa errada. Fiquei
muito grata quando descobri o conselho do Buddha nas escrituras
relativamente a dar \emph{feedback}. Era uma das coisas que eu tinha
tendência a evitar sempre que possível, mas um dos aspectos do nosso
treino monástico é aprender a sermos bons amigos entre nós e, quando
necessário, dar algum tipo de \emph{feedback}.

O Buddha mencionou várias coisas que devemos ter presentes quando damos
\emph{feedback}. Antes de mais devemos assegurar-nos de que é a altura
certa, o que significa criar o tipo adequado de contexto para a conversa
que é necessário ter. Para mim, quanto mais séria tiver de ser a
conversa, maior cuidado tenho, de forma a criar o ambiente correcto --
procurando uma situação na qual não haja interrupções e a outra pessoa
não seja distraída. Não vale a pena tentar dizer alguma coisa à outra
pessoa se ela está perturbada, ocupada ou com qualquer outra tarefa.
Desta forma, encontramos um momento no qual a pessoa nos dá toda a sua
atenção.

Enquando falamos com ela - falamos de forma suave e gentil, com um
coração bondoso, desejando-lhe verdadeiramente o melhor. Isto é muito
útil para ajudar a pessoa a descontrair, de forma a receber realmente
aquilo que temos para lhe dizer. Não vale a pena se a pessoa começa a
ficar defensiva, não vai ouvir seja o que for. De forma que falamos
suavemente, com um coração amável. Certificamo-nos igualmente que os
nossos factos estão correctos e falamos de forma clara, dando à pessoa
apenas a informação que necessita e não lhe dizendo uma porção de coisas
a mais que podem não ser úteis.

Estes são os pontos que o Buddha recomendou para se ter em conta nestas
situações. Ele não disse que não devíamos dizer nada a ninguém, quando
temos que dizer algo difícil de ouvir. Assim, se alguém está a fazer
algo danoso e perturbador e a causar uma série de problemas, não é caso
para apenas «deixar estar», e apoiar o discurso ou o comportamento
danoso - existem formas sábias através das quais podemos reagir. Acho
estes princípios muito úteis, tanto em termos de interacção com outra
pessoa como em situações mais generalizadas.

Estive recentemente numa conferência monástica na qual estavam monges e
monjas de diferentes tradições budistas. Como provavelmente têm noção, a
tradição Theravada tem tendência a ser bastante conservadora, em
particular no que diz respeito à~\mbox{relação} entre monges e monjas - os
monges são seniores e as monjas são juniores. Havia algum debate e
algumas pessoas expressavam a sua preocupação relativamente a isso. Um
monge encantador, que era na realidade muito júnior, disse: «Bem, se
alguma coisa está a causar sofrimento, então é bom fazer algo
relativamente a isso!» Foi muito bom ouvir isto. Tinha algo de
refrescante e sensível. Evidentemente, nem sempre as coisas são tão
directas, mas a ideia de que, se alguma coisa está a causar sofrimento
devemos, se possível, fazer alguma coisa relativamente a isso, é um
exemplo muito bom da Intenção Correcta.

No que diz respeito à nossa meditação, a questão era se é suficiente
apenas notar e reconhecer um obstáculo na mente. A minha resposta tem de
ser que reconhecer e aceitar esse obstáculo é um primeiro passo
importante, visto que, se reagimos com aversão, apenas tornamos o
obstáculo maior e mais forte.

Há alguns anos atrás a inveja era um problema para mim. Na realidade,
fui eu que criei o problema - desagradava-me tanto que o transformei num
problema terrível. Se já experienciaram inveja nas vossas vidas sabem ao
que me refiro. É tão desagradável que, na realidade, não querem falar
disso seja a quem for. Mas tem um efeito poderoso na forma como nos
relacionamos com os outros - não é fácil fazer de conta que não está
presente. Assim, eu tinha tornado esta inveja num monstro enorme.

Há uma história do tempo do Buddha sobre um monstro, um \emph{yakkha},
que é uma espécie de demónio. Este \emph{yakkha} foi para um dos reinos
celestes onde Sakka, um dos altos deuses, tinha o seu trono. Sakka, na
altura, não estava sentado no trono - tinha ido a qualquer sítio. Mas os
outros seres e servos que cuidavam dele estavam lá. Quando o
\emph{yakkha} entrou na corte, eles ficaram todos um pouco chocados.
Então o \emph{yakkha} pavoneou-se por ali dentro, dirigiu-se ao trono e
sentou-se nele. Houve um alvoroço: «Que atrevimento! Sai daí!» Mas o
\emph{yakkha} apenas se tornou cada vez maior. Os \emph{yakkhas} adoram
a aversão e o ódio - são coisas que os fazem ficar grandes e fortes.
Assim, ele estava ali sentado a olhar para todos e a sentir-se
fabulosamente bem. Então Sakka regressou e viu o que se estava a passar.
Ouviu o alvoroço, viu o demónio sentado no seu trono muito satisfeito
consigo próprio e, como um anfitrião elegante, foi até ao demónio,
sorriu e disse: «Muito prazer em vê-lo. Fico contente que se tenha
sentado no meu trono. Está confortável? Posso trazer-lhe mais umas
almofadas? Que tal uma chávena de chá ou de néctar? O que gostaria de
tomar?» Neste momento aconteceu algo estranho ao demónio: começou a
encolher. Tornou-se cada vez mais pequeno e mais embaraçado e
envergonhado. Então escorregou para fora do trono, curvou-se
humildemente perante Sakka e deixou a corte. Os cortesãos ficaram
pasmados. Não conseguiam de todo compreender o que tinha acontecido, mas
certamente que tinha resultado.

Assim, isto era o que eu precisava fazer relativamente à inveja! Antes
de compreender como praticar correctamente, eu tornava a inveja num
monstro, encarava-a como um problema enorme, em relação ao qual tinha de
fazer alguma coisa, em vez de a ver apenas como uma situação
impermanente que tinha surgido na mente. Sem dúvida que conseguia
reconhecê-la quando estava presente e, claramente, não era uma condição
agradável, mas não valorizava o facto de não ter de me identificar com
ela. Tudo o que era necessário era, simplesmente, reconhecer quando a
inveja estava presente e permitir que se alterasse. Ainda surge por
vezes - mesmo agora existem ocasiões em que me visita e eu \mbox{reconheço-a},
mas não faço dela um problema, por isso não permanece.

Acho que esta história é útil no que diz respeito a aceitar condições
negativas e difíceis que surgem. Por vezes simplesmente aceitá-las e
reconhecê-las é suficiente. É uma forma de libertar a aversão, a
negatividade. Ajahn Chah costumava falar muito acerca de largar, de
abrir mão. Podemos pensar que deixar ir é não ter, ou vermo-nos livres
de algo mas, na realidade, podemos apenas \emph{permitir} que vá embora,
em vez de nos agarrarmos ou tentarmos deitar fora. Tal como qualquer
outra condição que experienciamos: se surge, irá cessar. E, em geral,
cessa mais rapidamente se realmente a deixarmos partir. Por vezes as
pessoas usam a expressão «simplesmente deixar estar.»

Uma das razões que me levaram a escolher falar sobre este tema, tem a
ver com outra das perguntas que recebi - uma boa pergunta - que diz:
«Quando sentimos dor, tensão ou comichão, é melhor permanecer na mesma
posição e observar, ou podemos mexer-nos e coçar?» Penso que a minha
resposta tem de ser: depende. Se estão num retiro onde o professor diz
que têm de ficar quietos, então provavelmente é melhor seguir esse
conselho. Mas eu diria: «Porque não experimentar? Tentem diferentes
\mbox{estratégias}.» Podemos aprender muito ao observar pacientemente uma
comichão mas, por vezes, é bom termos um pouco de alívio.

