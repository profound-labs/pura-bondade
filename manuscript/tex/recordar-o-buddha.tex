\chapter[Recordar o Buddha, o Dhamma e o Sangha]{Recordar o Buddha, o Dhamma\newline e o Sangha}

É bom recordar as qualidades do Buddha, do Dhamma e do Sangha.

O Buddha é aquele que está desperto, alerta, atento à forma como as
coisas são, aquele que vê claramente a natureza da existência e não é
iludido pela aparência das coisas. Ele conhece a impermanência, a
(qualidade de) insatisfação e a impessoalidade de todo o universo
condicionado.

Os ensinamentos do Buddha são referidos como o Dhamma. Eles apontam para
a verdade da nossa existência e cada um de nós pode realizar esta
verdade, conhecê-la por si próprio. Podemos usar palavras, conceitos e
instruções para orientar a nossa atenção no sentido de saborear a
verdade. Esta é uma experiência directa. O Buddha disse que o Dhamma era
\emph{sanditthiko, akaliko, ehipassiko, opanayiko.} Estas palavras em
Pali traduzem-se como: «aparente aqui e agora», «intemporal»,
«encorajando à investigação» e «guiando para o interior». Podemos
saborear o Dhamma quando e onde quer que estejamos.

O próprio Buddha compreendeu o Dhamma há mais de 2.500 anos. Conheceu-o
directamente por si próprio e orientou os outros para que, eles também,
pudessem conhecê-lo por eles próprios. Homens e mulheres comuns
sentiram-se atraídos pelo Buddha e pelos seus ensinamentos, tendo-os
adoptado e aplicado nas suas próprias vidas. Depois, partilharam com
outros o seu conhecimento e os frutos da sua prática. Passaram este
conhecimento de geração em geração, até ao presente. A isto chama-se o
Sangha - a comunidade dos discípulos do Buddha. São descritos como
aqueles que praticam o bem, que praticam directamente, que praticam com
conhecimento, que praticam com integridade, que são sinceros
relativamente à sua prática e que experienciam os resultados, o
conhecimento que surge da prática e da aplicação dos ensinamentos.
Dizemos que este Sangha traz grandes bênçãos. Uma comparação que é usada
com frequência é a de um campo, um campo de bênçãos. O melhor tipo de
campo é aquele onde existe um bom solo, no qual podemos lançar sementes
e cuidar delas de forma a que cresçam. O Sangha pode ser comparado a um
solo fértil, onde as sementes podem enraizar.

No contexto de um retiro, desbravamos o solo. Desbravamos as distracções
desnecessárias através da prática da renúncia, ou da simplicidade. Pomos
de lado as nossas preocupações e actividades habituais, de forma a
podermos receber as sementes do Dhamma num campo limpo. Seguidamente,
através dos nossos esforços, cuidamos diariamente destas sementes,
regamo-las, certificamo-nos que têm claridade e luz do sol, de forma que
possam enraizar e crescer no solo fértil. Na natureza as coisas crescem
ao seu próprio ritmo, por isso temos de ser bastante pacientes. Tendo
estabelecido as melhores condições possíveis, confiamos que as sementes
irão ganhar raízes e crescer e que iremos experimentar e apreciar os
seus frutos. Temos de assegurar a remoção das ervas daninhas. Temos de
assegurar que protegemos as plantas e cuidamos delas, de forma a estas
crescerem fortes e saudáveis. Durante um retiro cultivamos atitudes
mentais positivas. Se notarmos pensamentos negativos que possam
desencorajar ou enfraquecer a nossa prática, \mbox{examinamo-los} atentamente
de forma a removê-los e permitirmos que as plantas cresçam sem os
obstáculos constituídos por estas causas prejudiciais.

A nossa prática traduz-se no cultivo da consciência momento a momento,
notando como as coisas são neste momento, reconhecendo como é o agora.
Se nos sentimos negativos, ou gananciosos, ou preguiçosos e sonolentos,
inquietos ou incertos, cultivamos uma atitude amigável relativamente a
esses estados. Depois dedicamo-nos a firmar e manter a consciência, ao
invés de nos sentirmos assoberbados ou de lutarmos com estes obstáculos.
Com consciência vamos directamente à verdade deste momento - como as
coisas são no momento presente.

Então podemos continuar a nossa prática, utilizando a respiração como o
foco da nossa atenção ou, se estamos muito sonolentos, mantendo uma
atenção vívida na postura e, se necessário, abrindo os olhos, de modo a
ficarmos completamente alerta, atentos como o Buddha, com o conhecimento
que o momento presente é assim, conscientes a cada instante.

