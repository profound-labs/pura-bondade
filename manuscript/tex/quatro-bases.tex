\chapter{As Quatro Bases da Consciência}

Gostaria de falar um pouco sobre as Quatro Bases da Consciência. Falamos
com frequência em «consciência» ou «presença». Nem sempre é claro como
atingir este estado, contudo, como decerto já se aperceberam, ele é
fundamental para o caminho da prática. Sempre que me colocam uma questão
ou alguém vem ter comigo para falar sobre algum problema que está a
viver, a minha resposta imediata é sempre «Consciência.» Evidentemente,
a maior parte das questões ou dos problemas das pessoas precisam de um
pouco mais do que isso. Este tipo de resposta não seria muito
satisfatório para elas! Contudo, é absolutamente fundamental começar por
estar presente naquilo que está a acontecer aqui e agora - e este é o
grande objectivo da nossa meditação: cultivar o hábito de presença.

Nos seus ensinamentos, o Buddha apresentou aquilo que denominou as
Quatro Bases da Consciência:

\begin{quote}
  kayanupassana: \emph{contemplação do corpo ou da forma}

  vedananupassana: \emph{contemplação dos sentimentos}

  cittanupassana: \emph{contemplação da própria mente}

  dhammanupassana: \emph{contemplação dos objectos-mente}
\end{quote}

São bases porque são pontos básicos de referência - estão sempre
presentes e constituem parte da nossa experiência diária, mas o nosso
hábito é concentrarmo-nos noutras coisas. Em retiro lembrem-se varias
vezes destes pontos de referência, estas bases para a presença, para a
consciência. Gradualmente, em vez de acreditarmos naquilo que passa pela
mente, tornamo-nos mais confiantes em firmar a consciência em torno de
uma ou mais destas bases, descobrindo que elas são mais fiáveis do que
todas as ideias que temos sobre nós próprios, sobre os outros, sobre a
nossa situação e por aí fora. Não que estas ideias e conceitos sejam
todos maus, alguns podem mesmo ser bastante úteis. Com efeito, são
essenciais para a vida em comunidade, em sociedade, mas há que os
compreender tal como são.

O corpo ou forma (\emph{kaya}), é a primeira base da consciência. O
Buddha utiliza muitas formas de reflectir sobre a experiência do corpo,
incluindo a consciência da respiração ou \emph{anapanasati}, e a
consciência da postura, quer estejamos a andar, de pé, sentados ou
deitados. O Buddha falou igualmente da consciência quando realizamos as
tarefas mais comuns, como comer, beber, debruçarmo-nos ou
espreguiçarmo-nos, de forma que, em qualquer momento do dia, possamos
notar o que o corpo está a fazer. Ele encorajou igualmente uma abordagem
analítica do corpo, contemplando-o sob o ponto de vista dos quatro
elementos - vendo os nossos corpos como constituídos de terra, água,
fogo e ar - ou reflectindo sobre as diferentes partes do corpo. Parte da
cerimónia de ordenação de monges e de monjas é levar a atenção para a
aparência exterior do corpo, a superfície do corpo. Assim, cada monge
noviço ou monja noviça repete a seguir ao professor:

\begin{quote}
  kesa \emph{(cabelo)}

  loma \emph{(pelos do corpo)}

  nakha \emph{(unhas)}

  danta \emph{(dentes)}

  taco \emph{(pele)}
\end{quote}

Outra coisa que recitamos no mosteiro é a «reflexão sobre as trinta e
duas partes», na qual trazemos à mente todas as partes do corpo: carne,
sangue, vísceras, pus\ldots{} torna-se bastante pormenorizada. O
objectivo desta prática não é fazer-nos sentir repulsa pelo nosso corpo.
Porém, é útil como um antídoto para sentimentos fortes de atracção
sexual. É uma forma de arrefecer as coisas mas, uma vez que constitui
uma prática muito poderosa, deve ser usada com cuidado, de preferência
com o apoio de um professor experiente, que nos possa ajudar a evitar o
desenvolvimento de sentimentos negativos relativamente ao corpo, e em
vez disso a alcançar um ponto de neutralidade, de desinteresse.

Existe uma história um pouco estranha da vida do Buddha relativamente a
esta prática. Depois de ter ensinado um grupo de monges a reflecção
sobre as trinta e duas partes, o Buddha partiu em retiro durante algumas
semanas. Enquanto esteve ausente suicidaram-se muitos monges, porque
ficaram repugnados com os seus próprios corpos e acharam que deviam
acabar com as suas vidas. O Buddha então enfatizou a importância de
equilibrar esta prática, encorajando a sua utilização apenas como um
antídoto para a atracção física, e que possa trazer-nos tranquilidade.

Dentro do mesmo espírito, o Buddha encorajou igualmente a contemplação
da morte, de cadáveres, utilizando isto para contrabalançar, de igual
forma, a tendência para nos preocuparmos excessivamente com o corpo.
Isto ajuda igualmente a aprofundarmos o nosso entendimento de que o
corpo é impermanente, apreciando as suas limitações. Pessoalmente, acho
que o resultado disto é aumentar um sentimento de admiração
relativamente à forma como o corpo funciona e sobrevive.

Assim, a contemplação do corpo é a primeira base da consciência. É uma
forma de estabelecer uma sensação de presença e ajuda-\linebreak-nos a compreender
e a largar os desejos, medos e anseios que podemos sentir em relação ao
corpo.

A segunda base da consciência é a contemplação das sensações. De facto,
em inglês «sensação» é, com frequência, utilizado como sinónimo de
«emoção», mas a palavra em Pali para «sensação», \emph{vedana}, não se
encontra directamente relacionada com emoção. \emph{Vedana} quer
simplesmente dizer sensações agradáveis, sensações desagradáveis ou
sensações neutras, nem agradáveis nem desagradáveis. Algumas emoções são
agradáveis e algumas emoções são sentidas como desagradáveis, mas
\emph{vedana}, a palavra Pali, está separada de emoção. Pode referir-se
a algo agradável que se sente tanto a nível físico como mental - tanto o
corpo como a mente podem ter sentimentos agradáveis, desagradáveis ou
neutros.

A sensação está presente em tudo o que experienciamos, apesar de, em
geral, praticamente não nos apercebermos dela, de não estarmos
verdadeiramente conscientes dela. A nível consciente, tudo o que sabemos
é que gostaríamos de ter mais determinadas coisas, de nos ver livres o
mais rapidamente possível de outras tantas, e no entanto existem ainda
outros aspectos dos quais nem sequer nos apercebemos. Assim, a
contemplação das sensações é uma base muito útil para a consciência.
Pode ajudar-nos a estar mais conscientes das sensações neutras, tal como
sentir a roupa no corpo, o ar na pele quando não está nem muito frio nem
muito calor, e também as ocasiões em que não estamos particularmente
incomodados ou perturbados seja com o que for. Todas estas são
experiências de sensações neutras, quer físicas quer mentais.

Acho igualmente útil a contemplação de sensações desagradáveis pois
ajuda-me a reagir à sensação de forma mais branda. Quando me sinto
aborrecida, perturbada ou confusa, tenho tendência a lutar com estes
estados se não estou plenamente consciente. Mas se recorro à
contemplação de \emph{vedana} - reconhecendo o aborrecimento ou a
confusão como uma sensação desagradável - notar apenas que esta é uma
sensação desagradável pode ajudar-me a permanecer presente com a
sensação. É surpreendente como, quando estamos presentes com a sensação
desagradável, ela se altera por si própria. Não temos de lutar com ela
para a tentar mudar. Na verdade, lutar normalmente torna-a muito, muito
pior, enquanto se apenas estivermos presentes com a sensação, esta irá
cessar por si própria.

Contemplar sensações desagradáveis no corpo é igualmente útil, porque
tendemos a reagir instintivamente a essas sensações. Contudo por vezes é
útil parar e reconhecer que, tudo bem, estas são sensações
desagradáveis, e assim podemos observá-las e \mbox{vê-las} a mudar. Pode ser
muito interessante se, digamos, temos uma comichão enquanto meditamos:
se a conseguimos suportar e apenas estar presentes com a sensação,
podemos notar como ela muda. Reparo que, por vezes, se fico presente com
uma comichão em qualquer parte do corpo, ela desaparece dessa parte e
depois tenho comichão noutra parte - pode ser muito divertido!

Contudo, podemos necessitar de responder de forma mais activa a (algums)
outros tipos de sensações físicas desagradáveis. Por exemplo, se temos
uma dor forte nos joelhos, por vezes descontrairmos à volta da dor, por
si só, permite que esta mude ou mesmo que pare por completo mas, outras
vezes, podemos necessitar de mudar de posição de forma a evitar lesões.
Se estamos doentes podemos necessitar de tomar medicamentos. Se a sala
onde estamos a meditar é demasiado fria ou demasiado quente, por vezes é
adequado simplesmente aguentar isso. Mas, por vezes, é apropriado vestir
mais roupa ou abrir uma janela.

Mas é importante estabelecer a consciência antes de fazer qualquer
destas coisas. Desta forma reagimos com sabedoria e compaixão, em vez de
apenas procurarmos algo que poderá trazer um alívio temporário mas que
poderá, a longo prazo, tornar as coisas muito piores. Estou a pensar em
particular em como as pessoas usam drogas ou álcool para lidar com as
sensações desagradáveis. Nem as drogas nem o álcool resolvem, na
realidade, seja o que for - em vez disso, provavelmente irão levar a
problemas mais graves.

A terceira base é a própria mente. Por vezes, comparo a mente a uma
sala. Temos uma sala, um espaço, e quando está escuro lá fora, a sala
está escura. Quando o sol brilha lá fora, a sala está iluminada. Pode,
por vezes, estar cheia de gente; outras vezes vazia. Da mesma forma, a
mente é como um recipiente que é afectado por várias coisas: pelas
nossas disposições, pensamentos e pelas nossas experiências. Assim, na
contemplação da mente, é-nos pedido que estejamos conscientes da mesma,
quando esta se encontra num estado descontraído e expandido e quando se
encontra num estado contraído. Podem já ter notado que, por vezes,
quando se sentem muito bem, existe uma sensação de expansividade na
vossa mente, mas se alguém vos diz qualquer coisa indelicada ou em
desacordo, a mente contrai-se imediatamente - torna-se muito, muito
pequena e apertada. Isto é apenas o que acontece; é apenas o que as
mentes fazem, não é por mal, é impessoal. Assim, esta é uma abordagem à
contemplação da mente ou da consciência.

O quarto princípio é \emph{dhamma}, que não é a mesma coisa que Dhamma
como refúgio, verdade. Mas, como a palavra \emph{nibbana}, que apenas
quer dizer «arrefecido» - como quando um fogo é extinto -, \emph{dhamma}
também tem um sentido muito comum e normal. Pode significar apenas
«coisa» ou «objecto». Assim, esta base refere-se à contemplação de
objetos-mente, o tipo de coisas, ou \emph{dhamma}, que acontecem na
mente. Por exemplo, podemos contemplar os pensamentos que estamos a ter,
ou examinar o nosso estado de espírito, em termos daquilo que são os
Cinco Obstáculos, notando se estão presentes ou ausentes. Os Cinco
Obstáculos, tal como referi anteriormente, são o desejo sensual, a má
vontade ou ódio, a preguiça e o torpor, a inquietação e a agitação, e a
dúvida. Quando estamos a ser incomodados por um dos obstáculos, por
vezes o simples facto de o notarmos e o nomearmos pode ajudar-nos a
parar de lutar com ele. Esta é uma forma de estabelecer a consciência
desse objecto mental em particular. Podemos igualmente utilizar a
contemplação de objectos da mente para reflectir sobre aspectos dos
ensinamentos e sobre como os aplicar à nossa situação.

Assim, a primeira base da consciência é a forma do corpo
(\emph{kayanupassana),} a segunda são as sensações
(\emph{vedananupassana}), a terceira é a mente ou a consciência
(\emph{cittanupassana)} e a quarta são os objectos-mente
(\emph{dhammanupassana).}
