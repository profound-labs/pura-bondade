\chapter{As Cinco Faculdades Espirituais}

Um dos apoios mais importantes para a vida religiosa e prática
espiritual, é ter bons amigos, a amizade espiritual. Assim, acho que o
mais difícil no final de um retiro, é quando as pessoas têm de partir e
regressar a uma situação onde, talvez, não existam outros que partilhem
o mesmo conhecimento e interesse. Em tempos, o Venerável Ananda falou
com o Buddha sobre esta qualidade da amizade espiritual, dizendo: «Esta
amizade espiritual é realmente maravilhosa. Deve equivaler, pelo menos,
a metade da vida religiosa.» Buddha disse: «Não a metade, é toda a vida
religiosa.» Precisamos de amizade espiritual para nos encorajar, apoiar
e ajudar a prosseguir na direcção correcta.

Alguns de vocês podem praticar em conjunto, em grupos ou em família, e
isso é algo no qual se podem regozijar. Contudo, mesmo que não vivam
próximo de outras pessoas que pratiquem, tendo passado este tempo em
conjunto em retiro, agora talvez tenham um sentimento de conexão com uma
rede mais alargada de pessoas. Pode igualmente ser útil pensar nos
mosteiros - no facto de existirem mosteiros na Grã-Bretanha e noutras
partes da Europa. Podem já ter tido oportunidade de visitar alguns
deles. Espero que todos tenham oportunidade, um dia, de visitar um ou
outro dos mosteiros, porque só por termos conhecimento que existem
pessoas que dedicam as suas vidas por completo à prática desta via pode
ser um apoio tremendo para nós.

Lembro-me da primeira vez que fiz um retiro com o Ajahn Sumedho - ele
estava acompanhado por mais dois ou três monges. Vi-os ali sentados na
nossa frente, com as suas cabeças rapadas e os seus hábitos. Falaram das
suas vidas e eu pensei: «Meu Deus, se eles estão dispostos a ir a estes
extremos - rapar a cabeça, usar hábito e seguir todas estas regras -
certamente que existe aqui alguma coisa.» Outro pensamento que recordo
dessa altura, ao escutar Ajahn Sumedho falar de uma forma muito
inspiradora, foi: «Mas \emph{eu} nunca conseguiria fazer aquilo.»
Todavia, ele pediu depois a cada um dos monges que estavam sentados com
ele que dessem uma pequena palestra de Dhamma e nessa altura pensei:
«Bem, se eles conseguem fazê-lo, eu também consigo.» Isso \mbox{deu-me} fé e
confiança para continuar a praticar.

Uma das coisas que compreendi que é importante na minha própria prática
é sustentar este sentimento de fé - fé que se trata de uma prática que
vale a pena e que dá bons resultados, e que também eu tenho a capacidade
para praticar e conseguir atingir estes resultados. No ensinamento sobre
a Génese Dependente, uma das ligações é que a fé surge do sofrimento.
Isto pode soar um pouco estranho mas, de certa forma, é óbvio: temos de
compreender que estamos doentes antes de nos sentirmos dispostos a tomar
um medicamento! Se nos apercebemos que estamos em sofrimento, que
necessitamos de ajuda, então vamos naturalmente estar interessados
quando temos uma oportunidade para escutar ensinamentos que nos
encorajam e inspiram - iremos ter a fé para agarrar os ensinamentos e
aplicá-los nas nossas vidas. Esta faculdade da fé, \emph{saddha}, é
extremamente importante na nossa prática. É uma das Cinco Faculdades
Espirituais, sobre as quais vou falar esta noite.

Quando vamos para um retiro podemos pensar: «Não sei se vou ser capaz de
fazer isto.» Não se tem fé suficiente e pode-se \mbox{ficar-se} tentado a ir
embora. Mas se as pessoas vêm ter comigo e dizem: «Ai, Irmã Candasiri,
acho que me vou embora.», digo-lhes: «Não, fique, vai conseguir!» Tento
dar-lhes algum encorajamento para lhes fortalecer a confiança, de forma
a que tenham confiança e fé suficientes para apoiar o aparecimento de
energia ou \emph{viriya}, que, como segunda faculdade espiritual,
possibilita que se desenvolva o esforço.

A fé, por si própria, não é suficiente. A fé é muito importante para nos
colocar no caminho correto, mas depois precisamos de nos aplicar, de
fazer um esforço, \emph{viriya}. Um retiro requer um esforço enorme:
sentarmo-nos durante períodos bastante longos e depois praticar
meditação a andar; levantarmo-nos cedo; praticar a contenção; manter o
silêncio; etc. Quando fazemos este esforço somos capazes de experienciar
alguns dos bons \mbox{resultados} da nossa prática, do nosso esforço. Tomar
nota destes resultados aumenta a nossa fé e isso faz com que nos
queiramos esforçar mais!

A terceira faculdade é a consciência ou \emph{sati}. Isto quer dizer
notar e estar consciente do movimento da mente e das alterações no
corpo, tornarmo-nos cada vez mais conscientes do que nos rodeia e das
outras pessoas e, de modo geral, tornarmo-nos mais sensíveis, mais
sintonizados com o que ocorre dentro de nós e ao nosso redor. Com
consciência podemos igualmente ser capazes de ter o discernimento de
saber como ajustar a nossa prática. Notamos se nos estamos a sentir
muito entorpecidos ou sonolentos e portanto levantamo-nos ou abrimos os
olhos. Se for apropriado, fazemos algum exercício vigoroso. Quando
praticamos por nós próprios, podemos pesquisar e encontrar coisas que
nos forneçam energia. Estas serão diferentes para cada um de nós. Alguns
de nós irão gerar uma grande energia e inspiração a partir da leitura e
do estudo, ou talvez da entoação de cânticos - cânticos de devoção ou
mantras - ou da prática de exercício físico, ou a falar com outras
pessoas sobre a prática ou a ouvir palestras e CDs, etc. É importante
que cada um de nós encontre, por si próprio, aquilo que sustenta a
energia e o entusiasmo pela prática.

É igualmente importante reparar na maneira como nos desencorajamos.
Tenho ocasionalmente falado de nos auto denegrirmos, de pensarmos mal de
nós próprios. Isto é algo que pode realmente roubar a nossa energia e o
nosso entusiasmo. Tenho muita sorte em ser monja há tantos anos. Nunca
quis \emph{não} ser monja, mas já atravessei tempos em que me senti
muito desencorajada com a meditação, permitindo-me mesmo pensamentos
como: «Detesto a meditação.» Mas, uma vez que sou monja, permaneci no
mosteiro, tinha de meditar e encontrei forma de o fazer.

Precisamos encontrar formas de nos encorajarmos e não sermos demasiado
exigentes connosco próprios, de ficarmos contentes apenas por estarmos
conscientes da mente e do corpo, tal como são. Dizer: «Tenho de ter um
\emph{samādhi} muito forte e tenho de me sentar a meditar todos os dias
durante três horas», é demasiado severo e exigente. Para alguns de nós
isso pode ser útil mas, para a maior parte, a nossa prática tem de ser
um empreendimento bastante mais modesto. Se a abordarmos de uma forma
mais modesta, com gentileza, apenas a senti-la, começaremos a gostar
muito da prática e anteciparemos com agrado as oportunidades de nos
sentarmos a meditar. E se sentimos que não nos conseguimos concentrar
muito bem, não fazemos disso um problema. Diferentes pessoas têm
capacidades diferentes. Algumas pessoas conseguem que as suas mentes se
concentrem com facilidade. Outras pessoas não o fazem tão bem, mas podem
ter outras qualidades importantes; podem ser muito generosas ou
compassivas, ou sábias.

As duas próximas faculdades espirituais equilibram as duas primeiras. É
como que um triângulo com a consciência no vértice superior: a
consciência monitoriza e olha pela forma como a nossa prática está a
correr.

As duas qualidades finais são \emph{samādhi} e \emph{paññā}.
\emph{Samādhi} é, com frequência, traduzida como «concentração», mas eu
prefiro a palavra «tranquilidade». \emph{Paññā} é discernimento ou
sabedoria. A sensação de tranquilidade, ou \emph{samādhi}, equilibra a
energia. Na nossa prática é importante reparar se somos do tipo
enérgico, que está sempre a fazer muitas coisas, ou se somos mais do
tipo de ter uma prática forte de \emph{samādhi}. Eu tendo a ser do
primeiro tipo. Gosto de fazer coisas. Gosto de ser útil.

É bom manter um equilíbrio entre estas duas faculdades, tanto na prática
da meditação como na nossa vida do dia-a-dia, de forma que a nossa
actividade seja desenvolvida com um sentido de consciência e de
respeito. Não fazemos necessariamente uma grande quantidade de coisas,
apenas por termos energia para tal. Sabemos como controlar o nosso
tempo. Precisamos equilibrar períodos de meditação com períodos de
actividade, ou períodos de descanso com períodos de actividade, de forma
a não ficarmos exaustos.

Algo interessante relativamente à meditação, que vi acontecer em
situações diversas, é que pode tornar-nos muito egoístas, muito
preocupados com a nossa própria prática. Lembro-me de há muitos anos,
antes de ser monja, estar num acampamento de verão para a prática
espiritual, nas montanhas. Nesse acampamento estavam pessoas que
gostavam muito de meditar e que se tornaram muito elevadas e refinadas,
e muito puras. Eram tão puras que não queriam lavar a loiça, não fosse
isso, de alguma forma, perturbar a sua energia. Mas quando praticamos
meditação precisamos cultivar um sentido de equilíbrio - nem demasiado
de uma coisa nem de outra. Assim, a nossa meditação influencia as nossas
actividades, e o sentimento de alegria que resulta dos nossos serviços
apoia a nossa meditação.

No que diz respeito ao outro par, fé e discernimento, já todos ouvimos
falar na expressão «fé cega», na qual continuamos a acreditar em algo ou
a desenvolver uma prática porque nos disseram que é bom para nós. Temos
fé nessa prática ou temos fé na pessoa que nos falou dela. Existem
algumas pessoas que utilizam a mesma técnica de meditação há vinte ou
trinta anos, sem reflectir se a mesma está a trazer algum benefício ou
não. Precisamos equilibrar a nossa fé com reflexão inteligente, com
discernimento. Se não estamos a ter grande benefício, precisamos
reflectir se estamos a praticar de forma correta.

Há um ensinamento do qual gosto muito, intitulado o \emph{Maha Mangala
Sutta}, o Discurso das Bênçãos Superiores. Este discurso consiste numa
série de versos que delineiam todas as coisas que trazem bênçãos às
nossas vidas, que proporcionam benefícios espirituais. Começa, de forma
interessante, com possuir amigos sábios, ter bons amigos, e evitar
pessoas insensatas e iludidas. Assim, ponderamos realmente as pessoas
com as quais nos associamos. Evidentemente, se se trata de um familiar
próximo que é insensato e iludido, provavelmente é bom passar algum
tempo com ele, porque talvez possamos ajudá-lo a ver as coisas de forma
mais clara.

À medida que praticamos, começamos a compreender gradualmente que os
nossos interesses se modificaram e, assim, temos menos interesse em
fazer as coisas que costumávamos apreciar. Isto pode ser bastante
doloroso, porque é como se tivesse ocorrido uma espécie de mudança e,
desta forma, já não há razão para estar com algumas pessoas com as quais
costumávamos conviver. No entanto, claro, surgem outras pessoas nas
nossas vidas. Fazemos amigos novos que têm interesses semelhantes e nos
podem encorajar naquilo que queremos realmente fazer.

Este é o primeiro verso do \emph{Maha Mangala Sutta}. Depois prossegue
por muitas coisas, aparentemente bastante vulgares, como ter um modo de
subsistência adequado, cultivar uma forma de falar bonita (Discurso
Correcto), cuidar da nossa família, evitar substâncias intoxicantes, ser
paciente, ter oportunidades de escutar e praticar o Dhamma e deter a
compreensão das Quatro Nobres Verdades.

Gosto muito do último verso. Este diz: «Apesar de viver no mundo, o
coração permanece imperturbável. O coração não treme; está liberto de
mágoa, confusão ou carência.» Quando me deparei com isto fiquei muito
entusiasmada, pois implica a possibilidade de manter esta
imperturbabilidade do coração enquanto se vive no mundo. Eu tinha estado
em retiros e praticado meditação, e certamente podia sentir um certo
grau de calma e imperturbabilidade quando estava em retiro, mas a
sugestão de que isto era possível mesmo fora da situação de retiro,
mesmo no meio de uma cidade movimentada, de um emprego difícil ou de
situações impossíveis de uma natureza ou de outra, com que
inevitavelmente nos deparamos nas nossas vidas neste reino humano,
entusiasmava-me muito.

Uma das coisas que as pessoas me dizem quando lhes pergunto sobre a sua
prática é: «A minha meditação ainda não é muito boa, mas reparo que
agora já não me aborreço tanto com as coisas que acontecem no meu
dia-a-dia.» No meu caso, quando comecei a praticar, estava interessada
na possibilidade de me tornar menos vítima do meu estado de espírito e
mais capaz de permanecer estável, mesmo quando as coisas são bastante
difíceis e tudo constitui um desafio. Era isto que o Buddha conseguia
fazer; evidentemente, todos nós temos ainda um pequeno caminho a
percorrer. Mas noto que estou muito interessada na possibilidade de me
fortalecer e ser mais capaz de permanecer estável quando as coisas são
difíceis.

Por vezes as pessoas pensam que viver num mosteiro deve ser muito
tranquilo e dizem: «Ah, eu adorava ir consigo e ficar no mosteiro, com
todos aqueles encantadores monges e monjas.» Bem, os monges e as monjas
são certamente encantadores mas todos nós somos igualmente seres humanos
e todos nos encontramos numa viagem para a libertação. Quando vivemos
numa comunidade, há alturas em que nos aborrecemos uns com os outros e
quando isso ocorre, não podemos ir a um bar e embebedarmo-nos ou ir à
discoteca e, dessa forma, deitar tudo cá para fora. Temos de ficar no
mosteiro e sentarmo-nos todos os dias no mesmo sítio, com as mesmas
pessoas. Por vezes, damo-nos muito bem e, por vezes, é horrível; mas
treinamo-nos para praticar a contenção e, assim, evitamos ser ofensivos
uns para com os outros, cultivando a qualidade de atenção passível para
se manter estável. Sempre que falhamos, sempre que não conseguimos
mantermo-nos estáveis, sempre que ficamos desequilibrados ou
perturbados, isso mostra-nos onde está a nossa próxima tarefa, onde
necessitamos de nos esforçar mais. É como um jogo ou um desporto
fascinante onde, a pouco e pouco, vamos desenvolvendo capacidades.

Assim, em vez de nos sentirmos desencorajados quando cometemos um erro,
ficamos interessados nele, como um novo desafio. Encontramos formas de
nos encorajar a nós próprios e de manter um sentimento de auto-respeito
e de dignidade. \mbox{Trata-se} de um caminho difícil, aquele em que todos
estamos, mas é um nobre caminho. É por isso que falamos das Quatro
Nobres Verdades, porque é necessária uma certa coragem para olhar para
as verdadeiras razões do nosso sofrimento. É necessária uma verdadeira
integridade para enfrentar o que poderão parecer fraquezas, falhas ou
medos. É preciso sermos muito honestos connosco próprios - e muitas
pessoas não estão dispostas a ser assim tão honestas.

Para finalizar vou recapitular as Cinco Faculdades Espirituais:

Fé (\emph{saddha}) -- fé em como vale a pena praticar, em como isso nos ajuda
e nos leva onde queremos chegar. E igualmente importante, senão mais
importante, é a fé em termos a capacidade de o fazer, apesar de poder levar
bastante tempo. Eu devia acrescentar que necessitamos igualmente de bastante
paciência.

Energia ou esforço (\emph{viriya}) -- quando temos fé, temos energia. Se nos
desanimamos a nossa energia desaparece. Assim, é verdadeiramente importante
sustentar a fé, de forma a suportar o esforço para nos aplicarmos no cultivo
da consciência, \emph{sati}.

Consciência (\emph{sati}) -- \emph{sati} olha pelas outras faculdades e
mantém-\linebreak-nas equilibradas.

Tranquilidade (\emph{samādhi}) -- esta reunião e concentração da atenção
constitui uma forma de equilibrar a energia.

Discernimento sábio (\emph{paññā}) -- a discriminação cuidadosa pode
equilibrar a fé.

Se nos aplicarmos e mantivermos a prática e o desenvolvimento destas
faculdades de uma forma equilibrada, eventualmente atingiremos um estado
no qual o coração não treme. Descobriremos que conseguimos manter esta
qualidade de consciência do momento presente, independentemente daquilo
que está a acontecer dentro de nós ou à nossa volta.

